\documentclass[12pt, a4paper, titlepage]{article}
%\usepackage{url}
% font size could be 10pt (default), 11pt or 12 pt
% paper size coulde be letterpaper (default), legalpaper, executivepaper,
% a4paper, a5paper or b5paper
% side coulde be oneside (default) or twoside 
% columns coulde be onecolumn (default) or twocolum
% graphics coulde be final (default) or draft 
%
% titlepage coulde be notitlepage (default) or titlepage which 
% makes an extra page for title 
% 
% paper alignment coulde be portrait (default) or landscape 
%
% equations coulde be 
%   default number of the equation on the rigth and equation centered 
%   leqno number on the left and equation centered 
%   fleqn number on the rigth and  equation on the left side
%
\usepackage[
  textwidth=17cm,
  outer=2cm,
  textheight=45\baselineskip,
  headheight=\baselineskip,
  includehead=false,% Default
  heightrounded,
]{geometry}
\usepackage{fancyhdr}
\usepackage{graphicx}\DeclareGraphicsExtensions{.ps,.eps}

\pagestyle{fancy}
\bibliographystyle{acm}
\fancyhead{}
\fancyhead[LO]{\leftmark}
\fancyhead[RE]{\rightmark}

\usepackage{float}
\usepackage{booktabs}
\usepackage{multirow}

\usepackage[hyphens]{url}
\usepackage{hyperref}
\hypersetup{breaklinks=true}

\usepackage{xcolor}
\usepackage{fancybox}

\usepackage[most]{tcolorbox}
\definecolor{ShadowColor}{RGB}{30,150,190}

%\usepackage[light]{draftcopy}

\providecommand{\versionnumber}{DRAFT 0.39}


\title{AppCoins\\ Distributed and Trusted App-based Transactions Platform}
\author{\small Paulo Trezentos  \\
  {\em  ISCTE / Aptoide}  \\
  \and 
\small  Diogo Pires \\
  {\em Aptoide} \\
  \and
  Aptoide Team
  }





\date{\today\\\normalsize Version \versionnumber} 
% \date{\today} date coulde be today 
% \date{25.12.00} or be a certain date
% \date{ } or there is no date 
\begin{document}
% Hint: \title{what ever}, \author{who care} and \date{when ever} could stand 
% before or after the \begin{document} command 
% BUT the \maketitle command MUST come AFTER the \begin{document} command! 
\maketitle


\begin{abstract}

%Today there are 2.1 bn smartphone users in the world and the number is expected to grow to 4 billion in 2020. To install apps and games, those mobile users use app stores. An app store is the distribution channel between the developer and the user. App stores generate more than \$77 billion yearly in gross revenue.  However, they seem to be incapable of solving the most basic issues that harm the app store experience: inaccessibility of in-app purchases, advertising model intermediation, malware and lack of innovation.
Today there are 2.1 bn smartphone users in the world generating more than USD77 bn in annual gross revenue. Those numbers are projected to double by 2020. However, app stores are still riddled with inefficiencies and malware. In-app purchases (IAP) are not accessible to the low-end market, in-app advertising is plagued by too many intermediaries, malware is still prevalent and innovation is slowing down.

\medskip

%The reasons are diverse: inadequacy of payment models to emergent countries and younger generations, no trust between the actors of the ecosystem and lack of standardisation defining clear interfaces and enabling market free entry for new players.
The reasons are diverse: payment models are not suited for emerging countries and younger generations; there is no trust between the actors of the ecosystem; and there is a lack of standardisation defining clear interfaces and enabling market free entry for new players.

\medskip

%AppCoins is an open and distributed protocol for app stores. It proposes to move to the blockchain three of the most critical flows of app stores: advertising, in-app billing and developer approval. By redesigning the transactions inside the app stores, it creates efficiencies by disintermediation and redistributes the value released in a way that create incentives for the AppCoins supported stores dissemination.
The AppCoins network is an open and distributed protocol built on the Ethereum blockchain. It aims to mitigate the current inherent deficiencies of app stores. By marrying blockchain technology with app store technology, app advertising, in-app billing and app-approval can be drastically improved and sped up through disintermediation and redistributing the unlocked value to end-users and developers.

\medskip

%The redesign of the app stores' processes is based on blockchain technology. An Ethereum ERC20 compliant token named ``AppCoin'' will be used by the developers to advertise their Apps to the users. From every advertising investment inside the app store, 85\% goes to the user. The user has to use those coins to buy items (in-app purchase) inside the apps and games, generating the return of the investment to the developers. In parallel, the Advertising and IAB transactions are used to establish the reputation of the developer.
The design of the AppCoins network rests on three main pillars: 1) transparency, 2) equitability and 3) community-focused. Firstly, open and transparent standards facilitate trust and privacy. Secondly, revenue shares are redistributed away from unnecessary intermediaries to end-users and developers. Thirdly, through open-source code, knowledge is accessible to the community.

\medskip

The current problems in the app stores flows are further described in this document, as well as the risks contained in each of them. Concepts like ``proof-of-attribution'', use Ethereum network smart contracts and state storage, allowing cryptographically to reach an acceptable digital agreement between users and developers.

%\medskip

%The design of the protocol received contributions of players in the app store industry as well as the blockchain community, aiming to change the app distribution in 3 ways: ``open'' with open standards that enables trust and privacy, ``circular'' where the value created in a transaction is not drained by intermediaries but flows in a virtuous loop, and ``shared'' where knowledge and open source code benefits every member of the community.

\medskip

%The bootstrap of the protocol will be accomplished in the next 12 months, using the 200 million yearly unique active users of the Aptoide app store user base, and leveraging the tokens not distributed in the ICO to incentivise developers, OEMs and users to use technology powered by the protocol.
This new AppCoins power app economy network will be launched within the next 12 months, leveraging on existing 200 million annual unique active users of the Aptoide app store. Token sale proceeds will be used to incentivise developers, OEMs and end-users. By 2022, Aptoide aims that 1.3 bn people use AppCoins powered app stores.  


%\medskip
%
%The protocol will reach the full maturity in 2022 when 1.3 bn people will use AppCoins-supported app store. 


\end{abstract}

%\tableofcontents % create a table of contens 

\tableofcontents


\section{Introduction and Problem Statement}

%XXX general idea. maybe introduce introduce a table of abbreviations?

% (Give a brief overview of App Store ecosystem and intermediaries.) 

% (Explaining the problem for each of the 3 main flows. Problem of double attribution, problem of refutation,....)

App stores are a distribution channel between the developer and the end user. Although software distribution exists since there is software development, the current model of smartphone became popular with the launch of Apple App Store in July 2008, and with its pre-load in iPhone 3G.

In the same year, but later in August 2008, Google announced the launch of Android Market\cite{wiki:market}, the App Store for Android.

These initial app stores followed a centralized model where one entity is responsible for assuring the core features of software distribution: file delivery, app discovery, financial transactions, and app approval. As the smartphone userbase grew, the centralized model started to show severe flaws. The flaws and problems identified are strongly related with the existent model: lack of trust and economical efficiency. 

By not being transparent, app stores don't earn the trust among the different stakeholders: developers, advertisers, users, and OEM manufacturers. Being centralized, they cannot benefit of a shared and crowdsourcing economy. Being closed source and hiding data, they don't promote competition and innovation.

% XXX not transparent: how?
% XXX hiding data: what data?

The App Coins protocol covers three core app store use cases:

\begin{itemize}
\item {\bf Advertising inside the app store}: the transactional flow where a developer pays for a user to install his app or game. There are different advertising models depending on the action that triggers the actual payment of the Ad: CPI (Cost per Installation), CPA (Cost per Action), CPM (Cost per thousand impressions),... There are different technology and platforms to support it: Ad networks, Exchanges and RTB (Real Time Bidding).
\item {\bf In-App Purchase}: when there is something that the user wants to buy inside the app or the game, like gems or unlock levels, the purchase mechanism is done through the app store. To enable those transactions the developer has to integrate the SDK from the App Store or to use the App Store API.
\item {\bf App approval}: in order for the app to be available the developer has to go through an approval process where the App Store screens the app through automatic tools like anti-virus, anti-malware tools, and static and dynamic code analysis platforms and then manually tests it.
\end{itemize}


\begin{figure}[!ht]
\centering
%\includegraphics[width=11cm]{diagrams/current_flows.eps}
\includegraphics[width=\textwidth]{diagrams/current_flows.eps}
\caption{Individual existent core flows in app stores.}
\label{fig:exist_flows}
\end{figure}




In the next sub-sections, we'll analyse each of the above flows and the main problems faced today.

\subsection{Advertising}


Currently, the three flows presented in figure \ref{fig:exist_flows} don't have any interaction among them. They are isolated and handled by different App Store teams. The resources and information generated by one flow are not reused by the others. The intermediaries are many and were introduced to solve the lack of trust and the need to integrate with different players in a fragmented market.  

% XXX which flows: the flows cannot be easily identified
% XXX The resources ... sentence seems to be redundant.


In the next sub-sections, we'll analyse each of the above flows individually and the main problems faced today.


For a developer or a publisher, the most natural place to advertise an application or game is where the users are looking for that kind of content: the app store.

\begin{figure}[!ht]
\centering
%\includegraphics[width=11cm]{diagrams/current_flows.png}
\includegraphics[width=\textwidth]{diagrams/cpi_flow.eps}
\caption{Cost per Installation (CPI) ecosystem.}
\label{fig:exist_flows}
\end{figure}

For simplicity, we will focus in the Cost per Install (CPI) model, since the difference to the other models, like Cost per Action (CPA) or Cost per thousand impressions (CPM), is a matter of who shares the risk and captures the value.

In an advertising model where the advertiser (developer) bids for an installation, we have the three different moments:


\begin{itemize}
\item {\em Campaign creation}: the developer (advertiser) defines the conditions for the ad to run in the store. Typically, he establishes a value for the bid representing the value that is willing to pay for an install. There are other types of conditions called ``filters'' and representing target requirements: it must run in a specific country, a specific smartphone, a specific operating system version,...
\item {\em Impression}: when the campaign conditions are met and the bid is competitive, the ad is shown. The user may click in the ad to see the complete description of the App.
\item {\em Install}: if the user installs the App, thus converting the impression, an attribution is due and the corresponding money is transferred.
\end{itemize}

% Create a itemize list above 

At each of the above moments, the lack of trust between the developer and the user carries different risks. Table \ref{tab:risks} summarizes the different risks at each stage.

% table

\begin{table}[ht]
\centering
\begin{tabular}{|l||l|l|l|} \hline
{\bf Role} & {\bf Campaign} & {\bf Impression}  & {\bf Install} \\ \hline
{\bf User} & & & Is not a real user \\ 
 & & & ({\em R1:Risk of fake person}) \\ 
 & & & Double conversion  \\
 & & & ({\em R2:Risk of double attribution}) \\
 & & & Don't use the app  \\
 & & & ({\em R3:Risk of no attention}) \\  \hline
{\bf Publisher  /}  & & & Selling the data to third-parties \\ 
{\bf App store} & & & ({\em R4:Risk of data leak})\\ \hline
{\bf Developer} & Not enough funds & Run out of & Don't pay \\  
 & to start campaign & budget & the conversion \\  
  & ({\em R5:Risk of default}) & ({\em R5:Risk of default}) & ({\em R6:Risk of repudiation}) \\  
\hline\end{tabular}
\caption{Risks in advertising industry classified by action and by role.}
\label{tab:risks}
\end{table}

The risks presented above are today managed in different ways by the advertising ecosystem and have different impacts:

\begin{tcolorbox}[enhanced jigsaw,sharp corners, drop fuzzy shadow=ShadowColor]

The {\bf\em R1.2:Risk of fake person} consists of the impression of the ad and later installation being presented to a non-real person (bot,...) with the purpose of deluding the advertiser. 


The {\bf\em R1.2: Risk of double attribution} happens with the possibility of the same user to count twice as a conversion, leading the developer to pay two times what was due.


The {\bf\em R1.3:Risk of no attention}  consists in the user install the app that is being advertised but not paying any attention to it. He may or not open the app, but not interact with the app, leading to a zero return-of-investment.


The {\bf\em R1.4:Risk of data leak} consists in the information regarding the user being leaked to third-parties for advertising purposes. Information about the user preferences is aggregated in Data Management Platform (DMP) platforms and later used by advertising in programmatic / RTB targeting. 

The {\bf\em R1.5:Risk of default} consists in the developer creating a campaign but not having enough funds to pay the conversions that are generated in that campaign. Therefore not paying the due amount.


The {\bf\em R1.6:Risk of repudiation} happens when the developer doesn't recognize the installation, failing to attribute the conversion to the publisher. The attibution is generally monitored by tracking platforms like AppsFlyer, Adjust or Kochava that have multiple variables that can be changed by the developer to define what it considers a real attribution. These variables can take in account the time window period between the click URL and the conversion, the network fingerprint, amoing others. Attribution, or the lack of attribution, is mining the industry with only 15\% to 25\% of the real installations being considered conversions.

%XXX what does mining the industry mean?
%XXX provide reference for the 15-25% statement

\end{tcolorbox}

These risks in the Advertising flow will be considered in a section ahead in the design of the AppCoins blockchain.


\subsection{In-App Billing}

% Description of In-App Billing inside the stores

In-App Billing (IAB), also called In-App Purchase, consists in the possibility for the user to buy digital items inside an app or a game. Although those items are perceived to be bough inside the App, the items are bought through the app store.

\begin{figure}[!ht]
\centering
\includegraphics[width=\textwidth]{diagrams/iab_flow.eps}
\caption{Current IAB flow and intermediaries.}
\label{fig:iab_flow}
\end{figure}


The need for the transactions for go through the app store were introduced as mandatory by Apple App Store and then by Google Play. The conditions for the developer's App to be distributed is that all the financial transactions have to be managed by the app store. 

The app store adds some value to this flow: 1) it may know the customer already and have is payments data thus easing the entrance hurdles for the user and providing a better user experience, 2) has the trust of the user where the developer may have not 3) develop the technology necessary, allowing the developer to focus in the app development.

Although IAB represents a market with a huge volume of transactions processed by Google Play and Apple, there is still two big challenges.

%XXX provide source for "huge volume of transactions by ..."

The number of users with a credit card loaded in the store is still a minority. Only small part of the world population has access to credit card. Alternative methods like pre-paid cards are an approach but they are physical and depend of points of sale, therefore don't scale well.

On the other hand, some of other payment methods like carrier billing have prohibitive margins that compromise the revenue share of 70\% for the developer. In some markets, the margin of the telecom operator varies between from 35\% to 60\% of the cost of the transaction. The reasons given by the telecom operators are: 1) high risk of fraud that has to be compensated 2) the users may cannibalize the telecommunications balance so the margin has to pay that possibility.

%XXX provide source for margin being between 35% and 60%

Having that the user has a proper payment method, there are still some risks that have to be mitigated:

\begin{tcolorbox}[enhanced jigsaw,sharp corners, drop fuzzy shadow=ShadowColor]

The {\bf\em R2.1: Risk of user data leak} consists in the information regarding the user being leaked to third-parties for advertising purposes. Information about the user preferences is aggregated in DMP platforms and later used by advertising in programmatic / RTB targeting.

%XXX R2.1 is copy and paste of R1.4!

The {\bf\em R2.2: Risk of digital goods lost} may happen when a user buys a digital good inside the game or app but it is not delivered. Often, the user doesn't have a way to recover the payment or claim the digital good.

The {\bf\em R2.3: Risk of double payment} occurs when the user pays twice for the same in-app item purchase. Also in this case, the user may not have a proof that he payed twice.

The {\bf\em R2.4: Risk of digital items cloning} when the user is able to duplicate and tranfer the digital good to another user, leading to losses for the developer that charge once for a digital good that is used twice.

\end{tcolorbox}

A platform that handles the IAB transactions has to deal with those risks.


\subsection{Apps Approval}

% Description of today state: central approval, manual QA, automatic QA, time taken, arbitrary approval

Apps approval is one of the more critical challenges of an App Store.

\begin{figure}[!ht]
\centering
\includegraphics[width=\textwidth]{diagrams/apps_approval_flow.eps}
\caption{App approval in centralized App Stores.}
\label{fig:app_approval_flow}
\end{figure}


By definition, the App Store is a channel between the developer and the user\footnote{This section was contributed by Joao Carneiro, Aptoide backend team member}.

% include contribution from João Carneiro
In order to enforce security, legal and business requirements, App Stores define limits in terms of acceptable app behaviour and/or content. These policies also mirror the store's philosophy (e.g. defining the acceptable content) and protect both users and developers against unwanted or potentially dangerous behaviour thus promoting trust. Policies can include general categories such as safety - protecting against malware behaviour, offensive content or physical harm - or legal - protecting privacy and intellectual property. More restrictive stores such as the Apple App Store also imposes strict rules regarding the user interface design, minimum functionality and quality. \cite{GooglePolicyWebsite} \cite{ApplePolicyWebsite}

%XXX I don't get the meaning of "general categories"

The risk of infringement occurs when new apps are added to the store. Therefore stores which are open to public upload of apps (e.g. Google Play Store or Apple Play Store allow submission by developers) need to ensure that uploaded apps abide by their rules by putting them through a reviewing process. The app approval flow is a critical process for App Stores as it ensures the compliance of new apps to the store's policy.

The app screening may be performed through manual and/or automatic processes and differ between stores as they are defined by their own policies. The manual process involves a group of people (typically belonging to the Quality Assurance and/or the Security Team) who manually install and test apps on real devices. They examine the apps' behaviour and content in order to decide whether each app respects the store's policy. The automatic process consists of a computer program which automatically analyses the submitted apps and compares features to a given dataset of rules, signatures, unwanted apps, content or behaviour. Multiple techniques may be used by the program to automatically classify given apps into unwanted, accepted or unknown states\cite{Bhattacharya2017}.

Google's Play Store and Apple's App Store, the current largest and most well-known app stores use a combination of both processes. When a new app is submitted to their store they first go through an automatic process which will automatically discard identified unwanted apps and then proceed to the manual process. However, the two stores differ in the techniques they use in their automatic processes and the amount of apps that go through manual reviewing\cite{AppleInsiderWebsite}\cite{AndroidWhitePaper}.

Apple App Store approval flow is simple. All submitted apps go through an automatic static analysis process, a method which examines the app code without running it. In this process, the apps are analysed for traces of calls to the Apple's private API as the company's policy only allows calls to their public API. The identified apps are discarded while all the remaining apps are passed on for manual reviewing \cite{AppleInsiderWebsite}\cite{AppleApprovalFortune}. Apple states the following most common reasons for failing their strict manual reviewing process: crashes and bugs, broken links, placeholder content, incomplete information, inaccurate description, misleading users, substandard user interface, advertisements, web clipping, similar apps and not enough lasting value \cite{AppleReviewRejections}. According to apple, the complete review process takes on average between 24 (50\%) to 48 hours (90\% of submitted apps)\cite{AppleReviewTime}. 

Google's Play Store has more evolved reviewing system where the automatic process involves a complex machine learning engine. This engine relies on multiple technologies including static and dynamic analysis (where both code and runtime behaviour is analysed), heuristic and similarity analysis (for finding new trends of unwanted apps) and signatures (identifying known unwanted apps). The engine also includes features from external independent security research as well as the developer's behaviour (history with other apps and billing profile) as well as meta-data such as ratings and downloads. The automatic process assigns to each application a risk level ranging from safe to harmful. Low risk applications are automatically accepted and high risk applications ae automatically rejected. Apps with medium risk level are submitted for manual reviewing\cite{AndroidWhitePaper}. According to publicly available information Play Store's reviewing process takes on average between 45minutes and 2 hours\cite{AndroidReviewTime}.

Although these app approving systems are capable of detecting a large number of unwanted apps they also pose problems to developers. Apple's automatic process has shown problems with false positives and rejecting legitimate apps \cite{AppleInsiderWebsite} and their strict policy is known to frequently changing the categories of rejected apps posing problems to developers of such apps. Also, Apple's reviewing process strongly based on human analysis is known to have flaws, namely not being able to detect apps which hide their malware behaviour being inactive for a given amount of time and showing a regular behaviour in order to escape the human test\cite{AppleFlaws1}. Other reports \cite{AppleFlaws2} have also shown a big presence of scamware in Apple's store where apps are able to scam users into paying for unneeded services. Google's more automated system has also been shown to have flaws \cite{AppleApprovalFortune} due to its more permissive system with apps being accepted without manual evaluation. Frequent security reports show breaches in the security control of Play Store reporting the existence of multiple malware (ransomware, backdoor, trojans) infected apps compromising several millions of devices and thus posing serious threats to users \cite{GoogleMalware1}\cite{GoogleMalware2}\cite{GoogleMalware3}.

\medskip

Both Apple's App Store and Google's Play Store have a history of refusing and banning apps. Examples of recent complaints include the rejection of the social network GAB's app by both Apple and Google\cite{AppRefusedGAB}, the refusal of music streaming Spotify app update\cite{AppleRefuseSpotify} or the Anti-Spam App\cite{AppleRefuseTRIAD} by Apple and the rejection of Popcorn Time, TubeMate, Adguard or Fildo by Google. \cite{GoogleBannedApps}. These rejections are however often considered unfair by developers who claim an abusive and anticompetitive behaviour as the apps conflict with other services provided by the app stores and have motivated complaints to legal authorities\cite{AntiCompetitiveClaim}.

\medskip

As described above, several risks can be found from the current centralized app approval processes:

\begin{tcolorbox}[enhanced jigsaw,sharp corners, drop fuzzy shadow=ShadowColor]

The {\bf\em R3.1: Risk of Malware} is the possibility of an app or game submitted to the app store being infected with a virus or a malware that steal data or damages the user device.

The {\bf\em R3.2: Risk of user data leak} consists in the information regarding the user being leaked to third-parties for advertising purposes. Information about the user preferences is aggregated in DMP platforms and later used by advertising in programmatic / RTB targeting.

%XXX again, R3.2 is just a copy and paste of R1.4

The {\bf\em R3.3: Risk of censorship} when an app store blocks the publishing of an app or game based in political, religious or social factors that are subjective and totally unrelated with technical aspects. The censorship can be self-inflicted when the company running the app store follows orders or guidelines of national governments or can be result of technical external restrictions that limits the access to the app store (great firewall of China, for example).

The {\bf\em R3.4: Risk of arbitrary decisions} happens when the app store denies the distribution of an app based in ``anti-competition claused''\cite{PlayTermsService} or other reasons only related of its business interest, even if the interest is in other markets or industries.

\end{tcolorbox}

\subsection{Paper organization}

This paper is organized in the following chapters. In this chapter we started to introduce the flows, the current challenges and the flaws they carry.

Chapter \ref{sec:design} will propose the overall design of the solution for the core app store flows supported by blockchain technology. 

Chapter \ref{sec:protocol} will deep dive in the blockchain technology, presenting the main data structures and algorithms that are proposed.

The current limitations of the blockchain technology when applied to app stores are introduced in chapter \ref{sec:limitations}.

In chapter \ref{sec:related}, related work that shares common approaches with App Coins protocol are introduced, as well as projects that inspired parts of the App Coins protocol.

% TODO replacing 6 by the \ref tag

The future protocol developments will be included in chapter 6 and this document will end with acknowledging the contributions of the several community members that contributed to this document with their suggestions and opinion.




\section{Design of the Solution}


% a diagram that integrates all the players 

\subsection{Elementary Components}

% (In-App Billing, Advertising, Reputation builder) 


\subsection{Requirements and Assumptions }

% A table 


\subsection{Protocol Overview}

%(Include a diagram with the players. Could be a sequence diagram as in Filecoin diagram or a component diagram)


\subsection{Payments trasactions}

% include the flows between wallets of payments and cativations


\section{AppCoins: Protocol Definition}
\label{sec:protocol}

As we have seen before, the AppCoins protocol addresses the use cases of Advertising, In-App Billing (IAB) and Developer Reputation within app stores by leveraging the blockchain construction to create value for the different participants.\\

In this section, we present the data structures and algorithms used to solve each of the aforementioned use cases. It will be organised as follows: a section for each core flow and subsections explaining the data structures, algorithms and wallets flows.

\subsection{Advertising}

Advertising campaigns in the context of a blockchain must be constructed in a way as to overcome the risks elaborated before.

\subsubsection{Use Case Flows}

Figure \ref{fig:ads_sequence_diagram} shows the interactions between the several parties that coexist in the Advertising use case.

\begin{figure}[H]
\centering
\includegraphics[width=\textwidth]{diagrams/ads_sequence_diagram.png}
\caption{Sequence diagram of the Advetising use case.}
\label{fig:ads_sequence_diagram}
\end{figure}



\noindent \textbf{Create Campaign (1)}. When developers (advertisers) want to create a campaign, they do so by creating an entry in the blockchain with all the required data to identify that campaign and the users that will be eligible to convert it. This data will contain information about the advertised package, the countries targeted by the campaign, the budget allocated by the developer to the campaign, the amount of \textsf{APPC} to be paid by attribution, the start and end dates of the campaign, and the IP validator of the users IPs. In addition, a campaign also contains information about the rules to be used in order to validate attributions to users. Also, when a campaign is created, the budget allocated to the campaign is transferred. That value is decreased by the amount of \textsf{APPC} sent to successful attributions. The campaign will remain active until there is no budget or the {\em end\_date} is reached. The way for a developer to cancel a campaign is to withdraw the existent funds. \\

\noindent \textbf{Gather Campaign (2) \& Install (3)}. When campaigns are saved in the blockchain, they will then be gathered by the app stores that match the filters of the campaign with their users. Upon matching, app stores then present the available campaigns to each user. It is important to mention that an app store should only show a campaign to a given user if and only if the user matches the campaign filters. If a user provides a \textsf{PoA} to a campaign with filters not matching the user's profile, the user will spend \textsf{APPC} to provide the \textsf{PoA} but will not be entitled to the attribution and the correspondent \textsf{APPC} because the provided \textsf{PoA} will be considered invalid by the protocol. In this case, the app store will see its earnings from the Advertising revenue share model lowered because users will realise the app store shows campaigns to users only with the intent of maximum profit. Users will then be compelled to move to other app stores to avoid the risk of losing the required amount of \textsf{APPC} when registering the campaigns \textsf{PoAs}.\\

\noindent \textbf{Trigger \textsf{PoA} (4)}. After the installation of an app by a user with a campaign associated with it, the SDK integrated in the app will trigger an AppCoins compliant wallet to compute the components of the \textsf{PoA}. It will trigger the wallet every 10 seconds during the 2 minutes needed to compute a valid \textsf{PoA}. For each trigger and for each consequent \textsf{PoA} component sent by the SDK, the wallet computes a \textit{nonce} in a Bitcoin-style proof-of-work (PoW) that serves to validate the component and help avoiding click spoofing. This process is detailed in Section \ref{sssec:ads_fd}. \\

\noindent \textbf{Compute \textsf{PoA} (5) \& Write \textsf{PoA} (6)}. After there are 12 components of a \textsf{PoA} for the same campaign computed within 2 minutes, the complete \textsf{PoA} is computed and written in the blockchain for registration and further validation. \\

\noindent \textbf{Validate \textsf{PoA} (7)}. The validation / invalidation of the \textsf{PoAs} is an optional part of the protocol where it is defined that a \textsf{PoA} is valid. If there is not an invalidation within 60 (sixty) minutes by the developer responsible by the creation of the campaign, the \textsf{PoA} is considered valid and the user is paid by the smart contract. Although it may carry a cost of the transaction fee, a quick validation by the developer would create a better user experience for the user, sine the reward would be attributed sooner.


\subsubsection{Data Structures}
\label{sssec:ads_ds}

\noindent \textbf{Campaign}. A \textit{Campaign} is a statement of intent to pay for the attention of users. Developers create \textit{campaigns} and submit them to the blockchain. App stores fetch open campaigns from the blockchain and propagate them to users. Campaigns are composed of an amount of funds, a duration, the amount of tokens per attribution and the filters (app name, app version, geolocation,...). Please refer to Table \ref{table: data_structures_ad} for more details. \\

\noindent \textbf{Advertising Ledger}. The \textit{advertising ledger} is the record of attributions, i.e. users that completed the required action (e.g. having an app open for at least 2 minutes) for a campaign. The ledger is populated in a way that avoids the \textit{double attribution problem}, i.e. attributions that come from different app stores for the same campaigns / apps in similar time intervals need to be identifiable across all the stores, as well as the users.

\begin{table}[H]
\footnotesize
\centering
\begin{tabular}{|p{.5\textwidth}p{.5\textwidth}|}
\hline
\multicolumn{2}{|c|}{Data Structures} \\
\hline \vspace{0.05cm}
\textbf{Campaign} & \vspace{0.05cm} \textbf{Advertising Ledger} \\
campaign $C_a : \langle F_{a}, \Delta t_{a}, T_{a}, filters, D_i \rangle$
\begin{itemize}
	\item Funds $F_{a}$, the amount of tokens the developer $D_{i}$ is willing to spend for $C_{a}$
	\item Duration $\Delta t_{a}$, the duration of $C_{a}$
	\item Tokens per attribution $T_{a}$, the amount of tokens to be sent from $D_{i}$ and distributed to the other parties per attribution
	\item Filters, the specifics of $C_{a}$, as the app name, app version, geolocation of $C_{a}$ and others available to $D_{i}$
	\item Developer $D_i$, developer that submitted a campaign $C_a$
\end{itemize} &
advertising ledger $L_{Ad} : (A^{1}_{t}..A^{n}_{t})$
\begin{itemize}
	\item Attribution $A^{i}_{t}$, $i$-th attribution in the advertising ledger $L$, which is composed as a mapping $A^{i}_{t} : \{C_{N}^{j} \to (U_{N}^{1}..U{N}^{n})\}$, where $C_{N}^{j}$ is the standardised campaign $C_a$ across all the app stores and $U_{N}^{i}$ is the normalised user $U_i$ across all the app stores
\end{itemize} \\
\hline
\end{tabular}
\caption{Data Structures for Advertising Use Case}
\label{table: data_structures_ad}
\end{table}


\subsubsection{Algorithms' Pseudo-code}

Table \ref{table: ads_use_case} presents in pseudo-code a more in-depth definition of the following methods. \\

\noindent \textbf{Create campaign}. When a campaign is submitted to the blockchain, the \textsf{CreateCampaign} method creates the campaign $C$ containing the all the parameters describing the campaign as the funds $F$, duration $\Delta t$, etc. In addition, the funds $F$ the developer wishes to allocate to the campaign are sent from the developer's wallet $W_D$ to the campaign's contract wallet $W_C$. Since the created campaign is in the blockchain accessible to every user, it is used to overcome the \textit{bid refutation} problem.\\

$\left[\begin{tabular}{p{.7\textwidth}}
\textsf{AD.CreateCampaign}
\begin{itemize}
	\item INPUTS:
	\begin{itemize}
		\item Campaign parameters:
		\begin{itemize}
			\item Funds $F$
			\item Duration $\Delta t$
			\item Tokens paid per attribution $T$
			\item Filters (geolocation, app name, app version,...)
		\end{itemize}
		\item Developer $D$
	\end{itemize}
	\item OUTPUTS: Campaign $C$
\end{itemize}
\vspace{-0.8cm}
\end{tabular}\right.$ \\

\noindent \textsf{Set attribution}. When a user has been attributed to a campaign $C$, i.e. the user performed the required action (e.g. had the app open for at least 2 minutes), \textsf{SetAttribution} checks the advertising ledger $L_{Ad}$ to make sure the user $u$ has not yet been attributed to the campaign $C$ and if so, the attribution is written in the ledger and each participant receives the correspondent tokens, i.e. the user, OEM and app store receive $T_u$, $T_{OEM}$ and $T_{AS}$, respectively. Since the method is constructed in a way such that the same user is unable to be attributed the same campaign in different app stores, it avoids the \textit{double attribution problem}. \\

$\left[\begin{tabular}{p{.5\textwidth}}
\textbf{AD.SetAttribution}
\begin{itemize}
	\item INPUTS:
	\begin{itemize}
		\item User $u$
		\item Campaign $C$
	\end{itemize}
	\item OUTPUTS: Result $R$ (0 or 1)
\end{itemize}
\vspace{-0.8cm}
\end{tabular}\right.$ \\

\begin{table}[H]
\scriptsize
\centering
\begin{tabular}{|p{.5\textwidth}p{.5\textwidth}|}
\hline
\multicolumn{2}{|c|}{Advertising Use Case} \\
\hline \vspace{0.1cm}
\textsf{AD.CreateCampaign}
\begin{itemize}
	\vspace{-0.3cm}
	\item INPUTS:
	\vspace{-0.4cm}
	\begin{itemize}
		\item Campaign parameters:
		\begin{itemize}
			\item Funds $F$
			\item Duration $\Delta t$
			\item Tokens paid per attribution $T$
			\item Filters (geolocation, app name,...)
		\end{itemize}
		\item Developer $D$
	\end{itemize}
	\item OUTPUTS: Campaign $C$
\end{itemize}
\begin{enumerate}
	\item Compute $C$ := \textsf{CreateCampaign}($F$, $D$, $\Delta t$, $T$, $filters$)
	\item Send $F$ from developer's wallet $W_D$ to campaign's wallet $W_C$
\end{enumerate} & \vspace{0.1cm} \textsf{AD.SetAttribution}
\begin{itemize}
	\vspace{-0.3cm}
	\item INPUTS:
	\vspace{-0.4cm}
	\begin{itemize}
		\item User $u$
		\item Campaign $C$
	\end{itemize}
	\item OUTPUTS: Result $R$
\end{itemize}
\begin{enumerate}
	\item Compute $InLedger$ := \textsf{CheckAdvertisingLedger}($u$, $C$)
	\item If $InLedger$ = 1:
	\begin{itemize}
		\item Set $R$ := 0
	\end{itemize}
	\item If $InLedger$ = 0:
	\begin{enumerate}
		\item Compute $TX$ := \textsf{Transaction}($u$, $C$)
		\item Compute $R$ := \textsf{WriteAdvertisingLedger}($TX$)
		\item Compute $(T_u, T_{OEM}, T_{AS})$ := \textsf{DivideTokens}($T$)
		\item Send $T_u$ to user's wallet $W_U$
		\item Send $T_{OEM}$ to OEM's wallet $W_{OEM}$
		\item Send $T_{AS}$ to user's wallet $W_{AS}$
	\end{enumerate}
\end{enumerate} \\
\hline
\end{tabular}
\caption{Advertising Use Case}
\label{table: ads_use_case}
\end{table}


\subsubsection{Wallet Transactions}

The transfers between wallets in the Advertising flow have to address some risks that were previously described in Chapter \ref{subsec:intro_ads}.

There is a wallet that contains the budget of the created campaign, which is meant to lock the budget, addressing the risk of default (R1.5) by the developer. To address the same risk, there is also a wallet that will temporarily store the value of the impression, to make sure that if the attribution occurs within a certain time, i.e. the user installs and opens the app within a certain time interval, there are funds available to pay for the conversion. Otherwise, there could be a scenario of serving an impression to many users but the campaign only had enough funds for one more attribution. The users would try to get attribution, creating a race condition, but only one would get it. An example of this schema is shown in Figure \ref{fig:wallet_cpi_flow}.

\begin{figure}[H]
\centering
\includegraphics[width=\textwidth]{diagrams/wallet_transfers.eps}
\caption{Wallet transfers in the CPI advertising flow.}
\label{fig:wallet_cpi_flow}
\end{figure}

\subsubsection{Messages Format Definition}
\label{sssec:ads_fd}

This section details the format of the objects needed for the Advertising use case. To implement the protocol one needs to be compliant with these messages to assure interoperability of the different components.

The messages and data structures exchanged are:

% insert diagram with structures
\begin{figure}[H]
\centering
\includegraphics[width=\textwidth]{diagrams/messages_exchanged_diagram.png}
\caption{Messages exchanged in Advertising usecase diagram.}
\label{fig:messages_diagram}
\end{figure}


Figure \ref{fig:ads_sequence_diagram} presents five different objects (data structures and interfaces):

\begin{enumerate}
\item Campaign definition (in the blockchain)
\item Attribution methods interfaces
\item PoA Trigger
\item PoA definition (in the blockchain)
\item PoA validation / invalidation definition (in the blockchain)
\end{enumerate}

\medskip

{\bf [1] Campaign definition (in the blockchain) }

When creating a campaign in step \textsf{(1)} from Figure \ref{fig:ads_sequence_diagram}, the advertiser (developer) specifies the filters that should be used by app stores to select users matching the defined criteria, as well as the validation rules that should be used when a \textsf{PoA} is submitted to verify if it complies with the campaign rules. Therefore, the campaign object has following format:
\begin{tcolorbox}[enhanced jigsaw,sharp corners, drop fuzzy shadow=ShadowColor]
\begin{lstlisting}[xleftmargin=0.05\textwidth]
{
    "campaign_id": "9e850e4ce590f20dc03568f144543741",
    "filters": {
        "country": ["US", "ES", "CA"],
	"package": "cm.aptoide.pt",
	"vercode": [4283,4284]
    },
    "attribution_validation_rules": {
        "vercode": "false",
        "IP_validation_required":"true",
        "country": "true",
        "IP_daily_conversions": 2,
        "wallet_daily_conversions":1
    },
    "price": 1,
    "budget": 10,
    "start_date": "2018-04-23 15:21:19+00:00",
    "end_date": "2018-04-25 15:21:19+00:00",
    "ip_validator": "whoami.developer.com",
    "dev_wallet": "0x59dc842b64c7229af88f8b0b02fdf04ff6f083ad"
}
\end{lstlisting}
\end{tcolorbox}

\medskip

{\bf [2] Attribution methods interfaces}

Regarding the attribution scheme, i.e. the mechanisms by which the SDK knows the wallet addresses of the app store and the OEM that should receive the revenue share attributed to app stores and OEMs for a given attribution, the protocol defines 3 different ways, with different degrees of priority, for the SDK to know the app store and the OEM that should receive the correspondent shares when a valid \textsf{PoA} is produced and sent to the blockchain. For all the 3 possibilities, there is a common component which is a JSON file stored by the wallet consisting of a mapping between app stores and OEM IDs and their respective Ethereum wallet addresses. This file has the following format:
\begin{tcolorbox}[enhanced jigsaw,sharp corners, drop fuzzy shadow=ShadowColor]
\begin{lstlisting}
{
  "stores": {
    "cm.aptoide.pt": "aptoide_address",
    "com.google.vending": "google_address",
    "default_address": "default_address"
  },
  "oems": {
    "xiaomi_oem_id": "xiaomi_address",
    "google_oem_id": "google_address",
    "default_address": "default_address"
  }
}
\end{lstlisting}
\end{tcolorbox} 

Regarding the first possibility, and the one with the highest level of priority, it consists in having the SDK asking the Android Operating System (Android OS) which app installed the one where the SDK is integrated. If the Android OS does not know because the app store, upon installing the app that integrates the SDK, did not inform it, then the second possibility is to have the app store broadcasting its ID and the OEM ID to the SDK. For these two options, when the SDK sends the \textsf{PoA} components to the wallet, it also sends the app store and OEM IDs, which should be mapped to the corresponding Ethereum wallet addresses using the aforementioned JSON file. Lastly, if the app store does not inform the Android OS that it installed the app nor this it send a broadcast upon installation, then the wallet should consider the default addresses for both the app store and the OEM that shall be attributed the share. 


\medskip

{\bf [3] PoA Trigger}


%%  This part is missing. We need to define the intent 

\medskip

{\bf [4] PoA definition}

Regarding the definition of the \textsf{PoA} components sent by the SDK to the wallet in step \textsf{(4)}, they hold data that enable the wallet to aggregate components by package. The format of each component is the following:
\begin{tcolorbox}[enhanced jigsaw,sharp corners, drop fuzzy shadow=ShadowColor]
\begin{lstlisting}[xleftmargin=0.05\textwidth]
{
    "ts": "2018-04-04 12:34:33+00:00",
    "package_name": "com.asfoundation.wallet"
}
\end{lstlisting}
\end{tcolorbox}

As said in step \textsf{(4)}, for each \textsf{PoA} component the wallet should compute a \textit{nonce} in a Bitcoin-style \textsf{PoW} that helps avoiding click spoofing by introducing a computation step that takes a certain amount of time. Given a \textsf{PoA} component as the one above, a \textsf{SHA256} hash $h_C$ is computed and the goal is to find an integer $nonce$ that when concatenated with the hash $h_C$ outputs another \textsf{SHA256} hash $pow$ with a certain number of leading zeros. As an example, assuming the \textsf{PoA} component above and denoting it as $C$:
\begin{tcolorbox}[enhanced jigsaw,sharp corners, drop fuzzy shadow=ShadowColor]
\begin{align*}
h_C &= sha256(C) \\
	    &= 2de1a25b781dda30fae45cb34495501c9fe3fb83f464e46c6d6e19b48f09b440 \\
n_{zeros} &= 5 \\
nonce &= 959928 \\
pow &= sha256(nonce + h_C) \\
		       &= 0000051232895be3f3a541907bcb5e69a87a4b39ebc3c43ee4d1f256b71c8364
\end{align*}
\end{tcolorbox}

After validating 12 \textsf{PoA} components, the wallet computes a full \textsf{PoA}, which is of the form:
\begin{tcolorbox}[enhanced jigsaw,sharp corners, drop fuzzy shadow=ShadowColor]
\begin{lstlisting}[xleftmargin=0.05\textwidth]
{
    "campaign_id": "9e850e4ce590f20dc03568f144543741",
    "poa_id": "x1g45gbhb2d26b0ef3cfbabc16f4553baa267byu",
    "ipvalidation": "signed_ip",
    "package_name": "com.asfoundation.wallet"
    "components": [
                    {"ts": "2018-04-03 16:39:27+00:00",
                     "nonce": 421},
                    {"ts": "2018-04-03 16:39:37+00:00",
                     "nonce": 1345},
                    {"ts": "2018-04-03 16:39:47+00:00",
                     "nonce": 5985},
                    {"ts": "2018-04-03 16:39:57+00:00",
                     "nonce": 3568},
                    {"ts": "2018-04-03 16:40:07+00:00",
                     "nonce": 7030},
                    {"ts": "2018-04-03 16:40:17+00:00",
                     "nonce": 6110},
                    {"ts": "2018-04-03 16:40:27+00:00",
                     "nonce": 2215},
                    {"ts": "2018-04-03 16:40:37+00:00",
                     "nonce": 1793},
                    {"ts": "2018-04-03 16:40:47+00:00",
                     "nonce": 6077},
                    {"ts": "2018-04-03 16:40:57+00:00",
                     "nonce": 1391},
                    {"ts": "2018-04-03 16:41:07+00:00",
                     "nonce": 149},
                    {"ts": "2018-04-03 16:41:17+00:00",
                     "nonce": 3688}
                  ]
}
\end{lstlisting}
\end{tcolorbox}
Note that each component in the full \textsf{PoA} includes the value of its corresponding \textit{nonce} for posterior validation, if needed. The \textit{package\_name} field is passed to the first level fields of the object since it is appears in every component. \\


\medskip

{\bf [5] PoA validation / invalidation definition}

%% This section is missing. Is very similar but it has to state what are the format for the invalidation messages in the blockchain

%% The invalidation has to be very clear about while rule was not match and identify why the attribution failed



\subsection{In-App Billing}
%missing intro?

\subsubsection{Use Case Flows}

Figure \ref{fig:iab_sequence_diagram} shows how the different parties taking part in the IAB use case interact.

\begin{figure}[H]
\centering
\includegraphics[width=\textwidth]{diagrams/iab_sequence_diagram.png}
\caption{Sequence diagram of the IAB use case.}
\label{fig:iab_sequence_diagram}
\end{figure}

There are several interactions taking place in the IAB use case, although only interactions \textsf{4}, \textsf{5} and \textsf{6} are expected to occur several times during the lifecycle of an app. \\

\noindent \textbf{Integration (1)}. This refers to the integration of the SDK in the app by the developer. During the SDK integration, the developer specifies the SKUs and related information, as well as the ETH account that shall receive the payments when users by in-app items. Therefore, the integration is very simple and does not require much implementation from the developer. \\

\noindent \textbf{Upload (2) \& Installation (3)}. When the developer finishes the integration of the SDK, the resulting APK needs to be uploaded to the app store, since the distribution of apps is done by the several app stores that integrate the AppCoins protocol. Users then install the apps from the app stores. At the moment of the installation, the app store communicates that the app was installed by it. This communication is key to make the app store eligible to take part in the IAB revenue share model and is done either by having the app store broadcasting this data to the SDK or by having it inform the OS of the device that it installed the app. From the app store ID now stored in the SDK, the mapping between this app store ID and the correspondent ETH account can be made by the wallet, since it stores a file with a mapping between the several AppCoins compliant app stores IDs and their ETH accounts. The file is a JSON file of the form:
\begin{tcolorbox}[enhanced jigsaw,sharp corners, drop fuzzy shadow=ShadowColor]
\begin{lstlisting}
{
  "stores": {
    "cm.aptoide.pt": "aptoide_address",
    "com.google.vending": "google_address"
  },
  "oems": {
    "xiaomi_oem_id": "xiaomi_address",
    "google_oem_id": "google_address"
  },
  "default_address": "default_address"
}
\end{lstlisting}
\end{tcolorbox} 

\vspace{0.2cm} \noindent \textbf{Payment (4)}. Whenever a user wants to buy an in-app item, the SDK triggers the payment flow. This payment flow is composed by a verification step to ensure an AppCoins compliant wallet is installed in the user's device and by a payment intent that should be caught by an AppCoins compliant wallet. If the no such wallet is installed, then user is directed to a place where it is possible to download one. An AppCoins compliant wallet is characterised by the implementation of the ERC-681\footnote{https://github.com/ethereum/EIPs/blob/master/EIPS/eip-681.md} with the format:
\begin{lstlisting}
ethereum:target_address?function=iab_pay(args)
args := (amount, prod_id, dev_addr, store_addr, oem_addr)
\end{lstlisting}

If there is a wallet installed that is listening to intents with the above format, the payment request is sent to it. The wallet is then responsible to process the intent and trigger the actual Ethereum transaction following the IAB revenue share model. \\

\noindent \textbf{Transaction (5)}. When the wallet receives the payment intent, the intent contains the amount of APPC the in-app item costs and its ID, and the ETH addresses of the developer of the app, the app store that distributed the app and the OEM that pre-loaded the app store. The wallet calls the smart contract that computes the revenue share for each of the players involved, triggers an event to store in the blockchain that the payment of the in-app item and then triggers the transfers for the developer, the app store and the OEM. \\

\noindent \textbf{Feedback (6)}. When the wallet triggers the payments, it then checks if they were accepted by the Ethereum network in order to give feedback to the app that the transactions are done. It then sends back the status of the transactions and their ID to the SDK in the app for further verification and acceptance.

\subsubsection{Use Case Format Definition}
\label{sssec:iab_fd}

\subsubsection{Data Structures}

\noindent \textbf{IAB Ledger}. The \textit{IAB ledger} is the record of items bought by users. It records transactions in a way that anyone can verify an anonymous user bought a quantity $Q$ of an item $I$ for a price $P$ from an app that integrated the IAB solution from app store $AS$.
\begin{table}[H]
\footnotesize
\centering
\begin{tabular}{|p{1.0\textwidth}|}
\hline
\multicolumn{1}{|c|}{Data Structures} \\
\hline \vspace{0.05cm}
\textbf{IAB Ledger} \\
IAB ledger $L_{IAB} : (TX_1..TX_n)$
\begin{itemize}
	\item Transaction $TX_i$, a transaction stating that an anonymous user $u$ bought a quantity $Q$ an item $I$ with price $P$ from an app $A$ that integrated the IAB solution from app store $AS$
\end{itemize} \\
\hline
\end{tabular}
\caption{Data Structures for IAB Use Case}
\label{table: data_structures_iab}
\end{table}


\subsubsection{Algorithms' Pseudo-code}

\noindent \textsf{Create transaction}. When a user $u$ wants to buy a certain amount $Q$ of items $I$, a transaction is created stating that the user $u$ bought a quantity $Q$ of an item $I$ for a price $P$ in a app $A$ that integrated the IAB solution from app store $AS$. \\

$\left[\begin{tabular}{p{.7\textwidth}}
\textsf{IAB.CreateTransaction}
\begin{itemize}
	\item INPUTS:
	\begin{itemize}
		\item User $u$
		\item Item $I$
		\item App $A$
		\item Developer $D$
		\item App store $AS$
		\item OEM $O$
	\end{itemize}
	\item OUTPUTS: Result $R$ (0 or 1)
\end{itemize}
\vspace{-0.8cm}
\end{tabular}\right.$ \\

\begin{table}[H]
\scriptsize
\centering
\begin{tabular}{|p{0.7\textwidth}|}
\hline
\multicolumn{1}{|c|}{IAB Use Case} \\
\hline \vspace{0.1cm}
\textsf{IAB.CreateTransaction}
\vspace{-0.3cm}
\begin{itemize}
	\item INPUTS:
	\vspace{-0.4cm}
	\begin{itemize}
		\item User $u$
		\item Item $I$
		\item App $A$
		\item Developer $D$
		\item App store $AS$
		\item OEM $O$
	\end{itemize}
	\item OUTPUTS: Result $R$ (0 or 1)
\end{itemize}
\begin{enumerate}
	\item Compute $TX$ := \textsf{Transaction}($u$,$I$,$D$,$AS$,$O$)
	\item Compute $R$ := \textsf{WriteIABLedger}($TX$)
	\item if $R = 1$:
	\begin{enumerate}
		\item App $A$ issues items to user
		\item Compute $(T_D, T_{OEM}, T_{AS})$ := \textsf{DivideTokens}($F$)
		\item Send $T_D$ to developer's wallet $W_D$
		\item Send $T_{OEM}$ to OEM's wallet $W_{OEM}$
		\item Send $T_{AS}$ to user's wallet $W_{AS}$
	\end{enumerate}
\end{enumerate} \\
\hline
\end{tabular}
\caption{IAB Use Case. In \textsf{CreateTransaction}, the issuing of items in certain app $A$ is purely done in the app based on the result of the method, since the items are not in the blockchain and there is no real blockchain transaction happening.}
\label{table: iab_protocol}
\end{table}


\subsubsection{Wallet Transactions}

In In-App Billing the transactions between wallets occur off-chain. The main transactions are between the user that is buying the digital item and 3 recipients: the developer, the OEM and the app store. Figure \ref{fig:wallet_iab_flow} presents the different flows of IAB transactions.

\begin{figure}[!ht]
\centering
\includegraphics[width=0.85\textwidth]{diagrams/wallet_transfers_iab.eps}
\caption{Wallet transfers in IAB flow.}
\label{fig:wallet_iab_flow}
\end{figure}


\subsection{Developer Rank}
\label{subsec:protocol_devrank}

There is the need to create trust between the different players in the app economy, namely between users, the developers and their apps. \\

In the AppCoins protocol, trust is characterised by a developer's rank, which is then propagated to all his apps. This rank can have the values of $\{"Unknown", "Trusted", "Critical"\}$. A user has the rank \textit{"Unknown"} only when joining the network for the first time, i.e. once the rank is changed from \textit{"Unknown"}, it can never have this value again. Changes in rank happen either through disputes, where the rank of the developer can change to \textit{"Critical"} in case of loss or remain the same in case of win, or by promotions, where the rank of the developer can change to \textit{"Trusted"}.

Promotions depend on the number of transactions in each of the developer's apps compared to the number of transactions in other popular apps. Promotions are automatic and depend on the following condition:

\begin{equation}
\sum\limits_{t=0}^{T} \sum\limits_{i=1}^{N} TX_{A_{ij},t} \geq TX_{A^{T}_{M}}
%T variable in the date and part of TX is confusing looks like T multiplied by X
\label{eq: promo_cond}
\end{equation}

Where $A_i = \{A_{i1}..A_{iN}\}$ is the set of N apps of developer $D_i$, meaning that $A_{ij}$ is the app $j$ in the set $A_i$, $t$ is the day with $t=0$ being the moment the developer joined the network, $T$ is the current day, $A^{T}_{M}$ is the app with the highest number of transactions on day $T$. Given these variables definitions, one can see that the left side in Equation \ref{eq: promo_cond} represents the amount of transactions in all the apps of developer $D_i$ since he joined the network and the right side represents the number of transactions of the app with the highest number of transactions on day $T$.

Contrary to the promotions, disputes are not done automatically and require explicit actions from users. Any user can open a dispute with a developer stating that the developer is dishonest. After the dispute is opened, any other user can join either side, depending on if they want to support the accusation of dishonesty or if they want to defend the developer.

Additionally, each rank value has a level associated with it. For \textit{"Unknown"} and \textit{"Critical"} rank values, the level is always set to 1. When the rank is \textit{"Trusted"}, we do not set a maximum level and it is expressed by:
\begin{equation}
S_l = \log_2 \frac{\sum\limits_{t=0}^{T} \sum\limits_{i=1}^{N} TX_{A_{ij},t}}{TX_{A^{T}_{M}}}
\label{eq: rank_level}
\end{equation}

Because of the use of the logarithm function in Equation \ref{eq: rank_level}, it becomes harder to gain higher rank levels as the rank level increases.

\subsubsection{Developer Reputation}

%A known developer is more trustable than a developer recently arrived to the apps' distribution business.
It is assumed that a known developer is more trustworthy than a developer who recently joined the apps distribution business.

\subsubsection{Data Structures}

\noindent \textsf{Dispute Intent}. A \textit{dispute intent} happens when a user claims that a developer is dishonest and no dispute against that developer is open. The developer or any other user then has 7 days to answer the dispute. If someone answers the dispute, be it the developer or any other user, the dispute is opened and the minimum fees needed to open it are captivated. If no user answers the dispute, the \textit{dispute intent} closes and the developer status changes to \textit{critical}. \\
%in the last sentence "If no user answers the dispute". Shouldn't it be "If no user or developer answers the dispute" else a user can allways blacklist any developer provided that other users don't reply

\noindent \textsf{Dispute}. A \textit{dispute} is a conflict between two parties, where one party - the \textit{contestants} - claim that a developer is dishonest, i.e. the developer uploads apps with malware, too many ads, or non-working apps and the other party - the \textit{pleaders} - claim the developer is honest. The pleaders include the developer being accused of dishonesty by the contestants. Both parties place tokens in the dispute and the party holding the most amount of tokens by the end of the dispute wins. Therefore, the \textit{dispute} includes the developer it regards to, the participants in both parties and their respective stake in the dispute. \\
%how many tokens each user, developer puts?

\noindent \textsf{DisputeLedger}. The \textit{dispute ledger} is the record of users joining disputes. It stores that a user joined a dispute on behalf of one of the sides with a certain stake (amount of tokens). The entries in the \textit{dispute ledger} are used to settle disputes when they end. \\

\noindent \textsf{RankLedger}. The \textit{rank ledger} is the record of rank changes of developers. Whenever there is a change in a developer's rank, be it an increase in the \textit{"Trusted"} rank levels or a change to \textit{"Critical"}, it is recorded in the \textit{rank ledger}.

\begin{table}[H]
\footnotesize
\centering
\begin{tabular}{|p{1.0\textwidth}|}
\hline
\multicolumn{1}{|c|}{Data Structures} \\
\hline \vspace{0.05cm}
\textbf{DisputeIntent} \\
dispute intent $K^{0}_{x} := \langle D_i, C_j, T_{min}, S\rangle$
\begin{itemize}
	\item Developer $D_i$, the developer being accused of being dishonest
	\item Contestant $C_j$, user claiming developer $D_i$ is dishonest
	%C has been defined as campaign
	\item Minimum fee $T_{min}$, the minimum fee needed to open the dispute that may result from this dispute intent
	\item status $S$, the current status of the dispute intent, which can take the values of \textit{"Open"} or \textit{"Closed"}
\end{itemize} \\
\textbf{Dispute} \\
dispute $K_x := \langle D_i, S, T_{min}\rangle$
\begin{itemize}
	\item Developer $D_i$, the developer being accused of being dishonest
	\item status $S$, the current status of the dispute, which can take the values of \textit{"Open"} or \textit{"Closed"}
	\item Minimum fee $T_{min}$, the minimum fee needed to open the dispute
	%who defines the value of $T_{min}$, dows it change between disputes?
\end{itemize} \\
\textbf{Dispute Ledger} \\
dispute ledger $L_{K} : (E_1..E_n)$
\begin{itemize}
	\item entry $E_i$, an entry containing information about a user joining a dispute in the form $E_i := \langle U_i, P, T, K_x\rangle$, where $U_i$ is the user, $P$ is the position the user is taking (can be either \textit{"Contestants"} or \textit{"Pleaders"}), $T$ is the stake (amount of tokens) the user $U_i$ is willing to use to defend position $P$ and $K_x$ is the dispute user $U_i$ is joining
\end{itemize} \\
\textbf{Rank Ledger} \\
rank ledger $L_{R} : (E_1..E_n)$
\begin{itemize}
	\item entry $E_i$, an entry containing information about a change in a developer's rank in the form $E_i := \langle D_i, S_b, S_a, S_l\rangle$, where $D_i$ is the developer, $S_b$ is the developer's rank before the change, $S_a$ is the rank after the change and $S_l$ is the level of the rank. $S_b$ can take the values of $\{"Unknown", "Trusted"\}$ and $S_a$ can take the values of $\{"Trusted", "Critical"\}$. For further details regarding the possible states of $S_b$ and $S_a$ and the possible values of $S_l$, please refer to Figure \textbf{[INCLUDE REF TO FIG]}.
\end{itemize} \\
\hline
\end{tabular}
\caption{Data Structures for Developers Rank Use Case}
\label{table: data_structures_da}
\end{table}


\subsubsection{Algorithms' Pseudo-code}

%user was previously $U$
%what are i,j,z,x ans 0
\noindent \textbf{Create dispute intent}. When a user $C_i$ claims a developer $D_j$ is dishonest, an intent of dispute is created, which may result in a dispute being opened, depending on whether someone answers the \textit{dispute intent} within 7 days or not. The user answering the dispute may not be the developer. \\

$\left[\begin{tabular}{p{.7\textwidth}}
\textsf{DR.CreateDisputeIntent}
\begin{itemize}
	\item INPUTS:
	\begin{itemize}
		\item User $C_z$
		\item Developer $D_j$
	\end{itemize}
	\item OUTPUTS: Dispute intent $K^{0}_{x}$
\end{itemize}
\vspace{-0.8cm}
\end{tabular}\right.$ \\

\noindent \textbf{Create dispute}. When a dispute intent is answered, a disputed is created. Within the following 30 days, any user may join the contestants side, which is composed by users claiming the developer $D_j$ is dishonest, or the pleaders side, which contains users stating that the developer $D_j$ is honest. The dispute intent that originated the new dispute is closed.\\

$\left[\begin{tabular}{p{.7\textwidth}}
\textsf{DR.CreateDispute}
\begin{itemize}
	\item INPUTS:
	\begin{itemize}
		\item Developer $D_j$
		\item Minimum fee $T_{min}$
		\item Dispute intent $K^{0}_{x}$
	\end{itemize}
	\item OUTPUTS: Dispute $K_x$
\end{itemize}
\vspace{-0.8cm}
\end{tabular}\right.$ \\

\noindent \textbf{Close dispute}. When the dispute is over (after 30 days), the winning side has its stakes refunded, while also receiving 10\% of each pledge from the losing side, with each winning member getting a winning stake proportional to their stake in the overall winning side pledge. Each member from the losing side gets a refund from the respective pledge subtracted by 10\%. Please refer to Table \ref{table: dr_protocol} for more details. \\

$\left[\begin{tabular}{p{.7\textwidth}}
\textsf{DR.CloseDispute}
\begin{itemize}
	\item INPUTS:
	\begin{itemize}
		\item Dispute $K_x$
	\end{itemize}
	\item OUTPUTS: None
\end{itemize}
\vspace{-0.8cm}
\end{tabular}\right.$ \\

\noindent \textbf{Compute rank}. The rank of the developer is periodically checked to assess changes in its value. For example, the rank can change from \textit{"Unknown"} to \textit{"Trusted"} or the \textit{"Trusted"} level can increase. \\

$\left[\begin{tabular}{p{.7\textwidth}}
\textsf{DR.ComputeRank}
\begin{itemize}
	\item INPUTS:
	\begin{itemize}
		\item Developer $D_i$
		\item Set of transactions in developer's apps $TX_i$
		\item Set of developer's apps $A_i$
		\item App with the highest amount of transactions $A_M$
	\end{itemize}
	\item OUTPUTS: None
\end{itemize}
\vspace{-0.8cm}
\end{tabular}\right.$ \\

\noindent \textbf{Record dispute}. When a user joins a side on a dispute, the event is recorded in the dispute ledger and can later be used to settle the dispute. \\

$\left[\begin{tabular}{p{.7\textwidth}}
\textsf{DR.RecordDispute}
\begin{itemize}
	\item INPUTS:
	\begin{itemize}
		\item User $U_i$
		\item Position $P$
		\item Stake $T$
		\item Dispute $K_x$
	\end{itemize}
	\item OUTPUTS: None
\end{itemize}
\vspace{-0.8cm}
\end{tabular}\right.$ \\

\noindent \textbf{Record rank change}. When there is a developer's rank change, it is recorded in the rank ledger. \\

$\left[\begin{tabular}{p{.7\textwidth}}
\textsf{DR.RecordRank}
\begin{itemize}
	\item INPUTS:
	\begin{itemize}
		\item Developer $D_i$
		\item New rank $S_a$
		\item New rank level $S_l$
	\end{itemize}
	\item OUTPUTS: None
\end{itemize}
\vspace{-0.8cm}
\end{tabular}\right.$ \\

\begin{table}[H]
\scriptsize
\centering
\begin{tabular}{|p{.5\textwidth}p{.5\textwidth}|}
\hline
\multicolumn{2}{|c|}{Developers' Rank Use Case} \\
\hline \vspace{0.1cm}
\textsf{DR.CreateDisputeIntent}
\vspace{-0.3cm}
\begin{itemize}
	\item INPUTS:
	\vspace{-0.3cm}
	\begin{itemize}
		\item User $C_z$
		\item Developer $D_j$
	\end{itemize}
	\item OUTPUTS: Dispute intent $K^{0}_{x}$
\end{itemize}
\begin{enumerate}
	\item Compute $K^{0}_{x}$ := \textsf{DisputeIntent}($D_j$, $C_z$)
	\item Set $K^{0}_{x}$.Status := \textit{"Open"}
\end{enumerate} & \vspace{0.1cm}
\textsf{DR.CreateDispute}
\vspace{-0.3cm}
\begin{itemize}
	\item INPUTS:
	\vspace{-0.3cm}
	\begin{itemize}
		\item Developer $D_j$
		\item Minimum fee $T_{min}$
		\item Dispute intent $K^{0}_{x}$
	\end{itemize}
	\item OUTPUTS: Dispute $K_x$
\end{itemize}
\begin{enumerate}
	\item Set $K^{0}_{x}$.Status := \textit{"Closed"}
	\item Compute $K_x$ := \textsf{Dispute}($D_j$, $T_{min}$)
	\item Set $K_x$.Status := \textit{"Open"}
\end{enumerate} \\ \vspace{0.1cm}
\textsf{DR.CloseDispute} 
\vspace{-0.3cm}
\begin{itemize}
	\item INPUTS:
	\vspace{-0.3cm}
	\begin{itemize}
		\item Dispute $K_x$
	\end{itemize}
	\item OUTPUTS: None
\end{itemize}
\begin{enumerate}
	\item Compute $WinSide$ := \textsf{WinningSide}($K_x$)
	\item Compute \textsf{DistributePledges}($K_x$)
	\item Set $K_x$.Status := \textit{"Closed"}
	\item If $WinSide$ = \textit{"Contestants"}:
	\begin{enumerate}
		\item Compute \textsf{DR.RecordRank}($K_x$.D, \textit{"Critical"}, 1)
	\end{enumerate}
\end{enumerate} & \vspace{0.1cm}
\textsf{DR.ComputeRank} 
\vspace{-0.3cm}
\begin{itemize}
	\item INPUTS:
	\vspace{-0.3cm}
	\begin{itemize}
		\item Developer $D_i$
		\item Set of transactions in developer's apps $TX_i$
		\item Set of developer's apps $A_i$
		\item App with the highest amount of transactions $A_M$
	\end{itemize}
	\item OUTPUTS: None
\end{itemize}
\begin{enumerate}
	\item Compute $S_l$ := \textsf{RankLevel}($TX_i$, $A_M$)
	\item If $S_l \geq 1$ and $D_i$.Rank = \textit{"Unknown"}:
	\begin{enumerate}
		\item Compute \textsf{DR.RecordRank}($D_i$, \textit{"Trusted"}, $S_l$)
		\item Return
	\end{enumerate}
	\item If $S_l > D_i$.RankLevel and $D_i$.Rank = \textit{"Trusted"}:
	\begin{enumerate}
		\item Compute \textsf{DR.RecordRank}($D_i$, \textit{"Trusted"}, $S_l$)
		\item Return
	\end{enumerate}
\end{enumerate} \\ \vspace{0.1cm}
\textsf{DR.RecordDispute}
\vspace{-0.3cm}
\begin{itemize}
	\item INPUTS:
	\vspace{-0.3cm}
	\begin{itemize}
		\item User $U_i$
		\item Position $P$
		\item Stake $T$
		\item Dispute $K_x$
	\end{itemize}
	\item OUTPUTS: None
\end{itemize}
\begin{enumerate}
	\item Send $T$ from user $U_i$ wallet $W_{U_i}$ to dispute's wallet $W_{K_x}$
	\item Set $E := \langle U_i, P, T, K_x\rangle$
	\item Compute \textsf{WriteDisputeLedger}($E$)
\end{enumerate} & \vspace{0.1cm}
\textsf{DR.RecordRank}
\vspace{-0.3cm}
\begin{itemize}
	\item INPUTS:
	\vspace{-0.3cm}
	\begin{itemize}
		\item Developer $D_i$
		\item New rank $S_a$
		\item New rank level $S_l$
	\end{itemize}
	\item OUTPUTS: None
\end{itemize}
\begin{enumerate}
	\item Set $S_b$ := $D_i$.Rank
	\item Set $E := \langle D_i, S_b, S_a, S_l\rangle$
	\item Compute \textsf{WriteRankLedger}($E$)
\end{enumerate} \\
\hline
\end{tabular}
\caption{Developers Rank Use Case}
\label{table: dr_protocol}
\end{table}


\subsubsection{Wallet Transactions}

The reputation of a developer is built based on blockchain transactions that can be associated to him. However, when there is a dispute, as presented in the previous section, the dispute mechanism is solved by receiving positive / negative endorsements of the community members to the developer. The final decision is based on the total sum of tokens that each side has. Figure \ref{fig:wallet_developers_approval} shows how users create disputes and join sides by placing tokens on the dispute.
%how many tokens each user and developer puts? Is it a fixed value? If variable, can rich malware developers or users takeover the system?

\begin{figure}[!ht]
\centering
\includegraphics[width=\textwidth]{diagrams/wallet_transfers_approval.eps}
\caption{Wallet transfers in the Developers Approval process.}
\label{fig:wallet_developers_approval}
\end{figure}


% descrever esta área após recebimeto do diagrama










\section{Blockchain Limitations and Proposed Approach}
\label{sec:limitations}


%The current status of blockchain developments, both Bitcoin and Ethereum, can limit the development of projects that depend on the technology to support the required business use cases. There are three use cases which the AppCoin protocol tries to solve in order to enable a fully working AppCoin economy:
With the current position of the blockchain industry, business use cases are reliant on technology supported by Bitcoin and Ethereum, which can delay the development of projects. There are three use cases which the AppCoin protocol tries to solve in order to enable a fully working AppCoin economy:

% \subsection{Use Cases}

\begin{itemize}
    \item Apps advertising in app store
    \item In-App Billing 
    \item Developer reputation leading to approval of their apps
\end{itemize}

From the aforementioned use cases a list of requirements on the technology can be established. 

\begin{itemize}
    \item Store digital currency / value securely
    \item Couple value storage and their transfers with logic
    \item Scalability for millions of users
\end{itemize}

The first requirement is solved by using a blockchain technology. Either Bitcoin or Ethereum work for this. For the second requirement only a type of blockchain technology supporting smart contracts will work, i.e. Ethereum, Nxt or Tezos.

\medskip

Unfortunately, neither basic Ethereum nor any other current blockchain for that matter are able to scale for the needs of an app marketplace. This problem is worsened further considering a platform that should potentially work for all the app marketplaces.

\medskip

First let us have a detailed look at the challenges in question.

\subsection{Blockchain limits}

Let us imagine a scenario of a working AppCoin economy using a blockchain technology like Ethereum. At the present there are about 2 billion Android users. The average Ethereum transaction is around 160 bytes \cite{EthereumTransactions}. If we wanted to create an AppCoin economy directly on Ethereum using these numbers, we would get a system with the characteristics shown in Table \ref{table:ethereumscalability}.

\begin{table}[!htbp]
\centering
\begin{tabular}{|l|r|}
\hline
Users              & 2 000 000 000    \\ \hline
Avg. daily TX      & 1                \\ \hline
Avg. TX size       & 160 bytes        \\ \hline
Daily TX           & 2 000 000 000    \\ \hline
TX per second      & 23 148           \\ \hline
Daily TX size      & 298.02 GB        \\ \hline
TX size per second & 3.53 MB          \\ \hline
TX size per month  & 8.73 TB          \\ \hline
TX size per year   & 104.77 TB        \\ \hline
\end{tabular}
\caption{Scalability requirements for AppCoin economy}
\label{table:ethereumscalability}
\end{table}

\medskip

{\bf Transaction Volume}

Scalability in terms of volume is the first problem of Ethereum. At present it handles only up to 20 transactions per second \cite{eth_scaling}. That is nowhere near the required 23 000 average transactions per second. At peak times the system should support a number of transactions several orders of magnitudes higher; for example VISA normally handles around 1667 transactions per second and at peak times it claims it can complete around 56 000 transactions per second \cite{eth_scaling}.

\medskip

{\bf Fees}

The topic of fees is closely connected to the possibility to support micro-transactions - transfers of value in the range of a few cents. At the time of writing the cost for one transaction in Ethereum is around 10 cents of a US Dollar for a transaction that takes around 1 minute to execute \cite{ethgasstation}. If the user is willing to wait around 10 minutes, the cost is 1 cent. This is both too slow and too expensive to facilitate micro-transactions.

\medskip

{\bf Latency}

As mentioned in the previous section, in Ethereum, one has to wait until a transaction is part of the blockchain. To have a reasonable guarantee that the transaction is part of the main blockchain and not in one of a number of possible branches, one needs to wait until 7 blocks have been created. This is because two miners could submit the next block at the same time and it will take some time to decide whose block will continue the chain. While this might not be an issue for high value money transfers in the AppCoin economy the transactions have to be executed within few seconds or faster (preferably instantly) in order to enable meaningful IAB and user experience.

\subsection{Existing technology}

Scalability restrictions are present in any blockchain technology using mining by \textit{proof-of-work}, such as Bitcoin and Ethereum. For Bitcoin - the oldest blockchain implementation - the proposed solution is the Lightning Network \cite{LighthingNetwork}.

%why is the lightning network the only solution referred? should talk about proof-of-stake?

\subsection{Ethereum and Bitcoin based}

Effectively there is one possibility for how to tackle the scalability problem with Bitcoin and two solutions with Ethereum. The first one is keeping the proof-of-work based blockchain and enhancing it with a direct-channel-payment solution like the Lightning Network. For Ethereum there are two major technologies that are going to be introduced also in this section.

\subsection{Lightning Network}
In brief the Lightning Network allows for creation of bidirectional payment channels that are off-chain \cite{LighthingNetwork}. Using such a channel two parties agree to deposit money into a common entity called a channel and this information is stored on the blockchain. There is a 2-signature entry for the channel with the total sum of the deposit saved. From this moment on the two parties can exchange between them any number of payments in the form of signed receipts. The only condition is that the balance for one party doesn't exceed the total of the deposit. Should a case like this happen, the parties can increase the deposit. These off-chain transactions are signed and sent direct. They are not broadcast. In contrast all the transactions on the main blockchain need to be broadcast to all participants. 

\medskip

This approach solves the blockchain scalability problem, such as:

\begin{itemize}
    \item The messages are direct and not broadcast. The volume of transactions between the two parties is only limited by their processing capabilities. Also, compared to blockchain, not all the transactions have to be stored but only the latest receipts, i.e. the latest balance for both users, must be stored instead of all transactions. This solves the problem of storing huge amounts of old transactional data that is inherent in blockchain.
    \item The only fees are for opening and closing the payment channel on the blockchain. For payments over the payment channel there are no extra fees. Once it is established micro transactions are possible without further limitations.
    \item Finally by using direct peer-to-peer (P2P) communication, transactions can be exchanged as fast as the underlying network between the two parties will enable it. Thus the problem of latency is solved.
\end{itemize}

Thus the original blockchain scalability limitations are resolved for the case of two directly connected parties. Of course having open channels between all the participants in the AppCoin economy is both not feasible and economically viable (because of opening and settling fees on the blockchain). Therefore the Lightning Network proposes a solution, where the payment channels are stored in a graph and payment can be routed securely keeping the above mentioned characteristics among multiple parties. This is enabled using hashlocks. For more detailed information, please have a look at either the specification of Lightning Network or the excellent introductory article \cite{starkness}.



\subsubsection{Raiden}
Raiden network is an early implementation of the Lightning Network protocol for Ethereum. Therefore, it is an off-chain scaling solution for Ethereum. It promises near-instant, low-fee and scalable payments and enables micro transactions \cite{Raiden}. Raiden runs as a network of nodes establishing payment channels among users. It uses locked funds in a smart contract in Ethereum and payments happen inside of Raiden until one party chooses to settle. Raiden will work for any Ethereum tokens implementing the ERC20 standard.

\medskip

One of the key characteristics of Raiden is its maturity. Currently it is the only known technology for Ethereum with existing implementation (i.e. beyond white paper stage) enabling a working App Coin economy. It has been under development for the past two years and it is reaching a Minimal Viable Product stage, currently being in a developer preview stage.

\medskip

Privacy is another key characteristic of Raiden.  All the transactions are known only to the involved parties and not public knowledge like in Ethereum.

\medskip

Finally the maturity can also be seen as a weak point for Raiden. Even though it is beyond proof-of-concept stage it is not recommended to be used on the live Ethereum network by its creators.

\subsubsection{Plasma}
Plasma is defined as a platform for "scalable autonomous smart contracts" \cite{Plasma} and it is another attempt to bring the Lightning Network technology from Bitcoin to the Ethereum world. This is achieved by means of a tree of hierarchical chains where most of the transactions can be settled in the sub-chains and do not need to go to the expensive and slow main chain. The main chain will be needed only to settle disputes coming from sub chains and would serve as final validation.

\medskip

A strong point for Plasma is that it is backed by key people from both Ethereum and Lightning Network, i.e. Vitalik Buterin and Joseph Poon. Another aspect supporting Plasma is that it has been endorsed by OmiseGO to be used for their decentralised exchange and payment platform\cite{omisego_plasma}. What speaks against Plasma at this time is the fact that the project has only a white paper, without any working code base yet.

\subsubsection{OmiseGO}
OmiseGO is an Asian payment gateway. It plans to use Ethereum and Plasma as a foundation for creating a decentralised exchange and payment platform \cite{OMG} and the company intends to launch its own blockchain called OMG that will be complementary to Ethereum. Ether will be then used as a medium of interchange for various assets being exchanged on this blockchain.

\medskip

Here are some of the strong points of OmiseGO:
\begin{itemize}
    \item Consensus rules for high performance activity;
    \item Rapid execution and clearing. Proof-of-Stake;
    \item Ability to handle extremely high volumes of transactions with final delivery in Ethereum;
    \item Availability as white label e-Wallet solution
\end{itemize}

The weak points include:
\begin{itemize}
    \item The project is at a very early stage. It was only announced at the beginning of August 2017. Currently, there is only a white paper available.
    \item As a result of the coupling with Ethereum, the OmiseGO white paper recommends to validate transactions simultaneously on the Ethereum blockchain for maximum security.
\end{itemize}

\subsection{Independent blockchains}
The second solution to achieve global consensus on scalability for blockchain is the use of a proof-of-stake as a block minting algorithm. For Ethereum this would be the Casper protocol which has been commonly talked about in the past years in the Ethereum community but is not implemented yet. As Casper is not in use yet we will examine here other projects that have implemented minting instead of mining blocks. A brief look will be also given to IOTA, which solves the scalability problem with a directed acyclic graph called Tangle instead of blockchain.

\subsubsection{Nxt}
One of solutions not based upon Ethereum is Nxt. It is described as "an open source cryptocurrency and payment network launched in November 2013 by anonymous software developer BCNext" \cite{Nxt}. Compared to Ethereum one can see Nxt as a monolith trying to include major features from all coins. It describes itself as an modern economic system based on cryptography and blockchain technology. On the other hand Ethereum built a low platform and programming language in which one can program generic contracts. Other major differentiation of Nxt to Ethereum is the fact that it is based on proof-of-stake for creating new blocks.

\medskip

Nxt is unsuitable for creating an App Coin Economy as it targets a different scenario. Focusing on managing a business, assets and customers and interaction between them are recorded in a secure way instead of a blockchain backed technology enabling users to create generic decentralised apps. Another point against Nxt is the fact that its beginning was obscure. 73 anonymous accounts received stakes in exchange for Bitcoin accounts and it is these people who can mint new blocks. And also that there has not been much buzz around and development for Nxt recently.

\subsubsection{Tezos}
Compared to Nxt the next examined project has received a great deal of attention recently and finished its ICO having raised more than \$200 million.

Tezos describes its ultimate aim to create blockchain technology working better than both Ethereum and Bitcoin. It wants to provide users with financial incentives for maintaining on-chain consensus mechanism. The main differences to Ethereum are:

\begin{itemize}
    \item Built-in consensus mechanism for governance.
    \item Decoupled from Ethereum. Separate blockchain with proof-of-stake consensus for block minting.
    \item OCaml as programming language for creating smart contracts.
    \item Design-wise Ethereum is a generic and low-level solution while Tezos aims to provide a fat protocol with a lot of features \cite{TezosEth}, not unlike Nxt.
\end{itemize}

\subsubsection{IOTA}
The goal of IOTA is to create a lightweight network that allows a machine economy. A machine economy is a situation where IoT (Internet of Things) connected devices communicate and execute payments between each other. IOTA does not use blockchain but a directed acyclic graph called Tangle. Currently it is in Beta stage with a reference implementation. One of the major advertised features is that there are no fees. In Tangle, each new transaction confirms at least 2 previous transactions; that is the "fee" to participate. This might delay the transaction, especially if the network is small. There are no miners. The IOTA white paper states it is quantum proof (invulnerable to encryption-breaking attacks), that it has a high level of supply and the tokens are not divisible, unlike Bitcoins or Ether. For a primer on IOTA see \cite{IOTA}.

\subsection{Proposed Approach}


\begin{figure}[!ht]
\centering
\includegraphics[width=\textwidth]{diagrams/offchain_wallets.eps}
\caption{On-chain and off-chain transactions.}
\label{fig:offchain}
\end{figure}


As seen in the previous section, there are blockchain currencies that can securely store digital value. But as we have seen, doing transactions over them does not scale, is slow and requires fees. Therefore alternatives like direct payment channels have to be evaluated.

\medskip

In general terms, the App Store acts as the gateway to the digital payment world for the user. He deposits his valuables (preferably Ether) with the store in a payment channel using a smart contract. The same happens for developers. And from this moment on they can utilise the micro-payments within the App Infrastructure. The payments are instant, cheap and there can be any number of them.

\medskip

The user has to pay fees on each Ether deposit to his account with the App Store and on checking out, i.e. settling his payment channel. Further fees may be the taxation done by the App Store used to pay for its infrastructure.  If there are several App Stores in the same ecosystem like we envision, there might be further (minimal) fees to forward the payments from one to another.

\medskip

Figure \ref{fig:offchain} depicts a schema of the wallets on-chain (Ethereum network) and the wallets off-chain using Raiden or OmiseGO.

% TODO explain better the division

\medskip

Returning to the requirements from the beginning of this chapter, we plan to use Ethereum for storing value and smart contracts. To have scalability, Raiden is being evaluated for the first Beta version of the reference implementation of the protocol that is due six months after the ICO. %could not understand sentence

\medskip

The choice for Raiden is because - at the time of writing - it is the only platform that has a test network and can be used. Yet OmiseGO has a promising outlook and in the future could be adopted depending on its progress. The company will be kept under close monitoring as it might develop the potential to substitute Raiden.






\section{Related Work}

This section presents the projects that inspired the AppCoins protocol solution to some extend, either because of the technology employed or by presenting concepts that underly the ones in our proposal. We first give a brief overview of each project and after we explain how each of our uses cases benefits from the contributions of each of the projects.

\subsection{Related Projects}
\subsubsection{Basic Attention Token}

The BAT project aims to revolutionise the digital advertising landscape by proposing a "decentralised, transparent digital ad exchange based on Blockchain" \textbf{[REF BAT PAPER]}. Their proposal is constituted by two components:
\begin{itemize}
	\item Brave: a browser that blocks third-party ads and trackers, which decreases webpages load time and ensures anonymity, while also building a ledger system that tracks users attention to ads in order to correctly reward publishers and advertisers
	\item BAT: a token for the decentralised ad exchange, connecting advertisers, publishers and users, while rewarding users for their attention
\end{itemize}

In their proposal, they want to eliminate the middlemen between advertisers and publishers, pay for user attention instead of CPM/clicks, provide faster webpage loads and ads tuned to user preferences, amongst other advantages. \\

By taking out the middlemen, they avoid the draining of resources by agencies, DSPs, exchanges, ad networks, and others, while also eliminating part of the complexity of having to handle with this huge ecosystem that is in place today. The gain in resources by eliminating the resource-draining players enables the sharing of resources by the important players in the flow: advertisers, publishers and users. Since there are less resources being wasted in middlemen, there is more available to be employed in processes that increase the value to the end user. They also propose to use machine learning at the browser level to serve tailored ads to users, instead of serving ads with no value. In addition, users are rewarded by their attention, which the project states as the "rare quantity", since the information available is far greater that the available attention each user has to give. \\

Today, publishers are paid based on clicks on ads. BAT proposes to start rewarding publishers based on the attention users give to ads, by keeping track of the user attention on a ledger system implemented in Brave, while always maintaining users' anonymity. User attention, as a very valuable asset, is not being rewarded correctly and users do not get anything while they navigate webpages and see the ads. The solution proposes that users also start receiving rewards for time spent seeing the ads while navigating, based on the amount of time they spend looking at them. \\

In order to reduce fraud, they propose approaches - calling them basic attention metrics (BAM) - to correctly identify users paying attention to ads. When the user attention is identified, it is saved in an anonymous way, while also ensuring that users do not get rewarded by paying attention to the same ad more that once. They define a \textit{proof-of-attention}, which ensures that a user can only see and get attributed to an ad once and maintains users' anonymity. This is achieved by using \textsf{ANONIZE} \textbf{[REF ANONYZE PAPER]} algorithm in a first stage. According to the authors, the algorithm is "the first implementation of a provably-secure multiparty protocol that scales to handle millions of users".  The BAT team says that they may also invest into using algorithms such as BOLT \textbf{[REF BOLT PAPER]}, zkSNARKs \textbf{[REF SNARKs PAPERS]} and STARKs \textbf{[REF STARKs PAPER]} to protect users' privacy.

\subsubsection{Kin}

The Kin project intends to create the "first open and sustainable alternative ecosystem of digital services
for our daily lives" \textbf{[REF KIN PAPER]}. In order to achieve this, a new cryptocurrency Kin is created to be used within the new ecosystem by the digital services. \\

Since Kin is being developed by Kik, a popular chat app with already millions of users, Kik will integrate Kin to showcase the possibilities of having an ecosystem of connected digital services. New partners joining the ecosystem will create a network effected, boosting the value of Kin. Kik will develop two main components of the new ecosystem:
\begin{itemize}
	\item Kin Reward Engine
	\item Kin Foundation
\end{itemize}

The Kin Reward Engine is going to create incentives for other digital services to adopt Kin. The majority of the Kin supply will be allocated to Kin Reward Engine and periodically will unlock and distribute a certain amount of Kin amongst the digital services within the ecosystem. The amount each digital service receives depends on the amount of Kin used by them. \\

Kin Foundation is the entity that will oversee the growth of the ecosystem, as well as administer the Kin Reward Engine. In time, the Kin Foundation will transition the entire ecosystem, including the Kin Reward Engine to a fully decentralised and autonomous network. When this happens, Kin Foundation main responsabilities will be helping onboarding new partners and overseeing development of fundamental components such as identity and reputation management, cryptocurrency wallets, and compliance solutions. \\

In the end, Kin wants to develop an ecosystem that is open and fair, where users benefit from a vast and diverse digital experience, being able to transition between services with almost no effort. Providers will be able to compete for compensation within the ecosystem.

\subsubsection{Monetha}



\subsection{Projects Contributions}
\subsubsection{Advertising}

\subsubsection{IAB}

\subsubsection{Developers Reputation}








% \section{Acknowledgements}
\label{sec:acknowledgements}

The AppCoins protocol had the contributions of various people. Apologies if we unfairly leave someone out of the deserved credits.

The Client-side support, section \ref{sec:clientside}, was a contribution of Marcelo Benites. The app approval, section \ref{subsec:intro_approval}, was written by Jo\~ao Carneiro. The Blockchain limitations section was contributed by Martin Uzak. Matthew Boyle reviewed the document for language inconsistencies. The rest of Aptoide team provided important feedback and ideas that were integrated in the protocol and in the text.

The following people provided much appreciated suggestions and corrections: Jonathan Becker and Thomas Gieselmann (e.Ventures), Jun Hasegawa (OmiseGO), Mathias Grønnebæk (Braveno), and Alexander Maier (ICO Advisories).


%\section{Future Work}

%(where we can include what is not yet addressed and how we see the evolution)


\section{Acknowledgements}

% (Where we give credit to other people in the team and externaly that contributed to the document)

\bibliography{appcoins_whitepaper}

\end{document}
