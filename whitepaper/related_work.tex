\section{Related Work}

This section presents the projects that inspired the AppCoins protocol solution to some extend, either because of the technology employed or by presenting concepts that underly the ones in our proposal. We first give a brief overview of each project and after we explain how each of our uses cases benefits from the contributions of each of the projects.

\subsection{Related Projects}
\subsubsection{Basic Attention Token}

The BAT project aims to revolutionise the digital advertising landscape by proposing a "decentralised, transparent digital ad exchange based on Blockchain" \textbf{[REF BAT PAPER]}. Their proposal is constituted by two components:
\begin{itemize}
	\item Brave: a browser that blocks third-party ads and trackers, which decreases webpages load time and ensures anonymity, while also building a ledger system that tracks users attention to ads in order to correctly reward publishers and advertisers
	\item BAT: a token for the decentralised ad exchange, connecting advertisers, publishers and users, while rewarding users for their attention
\end{itemize}

In their proposal, they want to eliminate the middlemen between advertisers and publishers, pay for user attention instead of CPM/clicks, provide faster webpage loads and ads tuned to user preferences, amongst other advantages. \\

By taking out the middlemen, they avoid the draining of resources by agencies, DSPs, exchanges, ad networks, and others, while also eliminating part of the complexity of having to handle with this huge ecosystem that is in place today. The gain in resources by eliminating the resource-draining players enables the sharing of resources by the important players in the flow: advertisers, publishers and users. Since there are less resources being wasted in middlemen, there is more available to be employed in processes that increase the value to the end user. They also propose to use machine learning at the browser level to serve tailored ads to users, instead of serving ads with no value. In addition, users are rewarded by their attention, which the project states as the "rare quantity", since the information available is far greater that the available attention each user has to give. \\

Today, publishers are paid based on clicks on ads. BAT proposes to start rewarding publishers based on the attention users give to ads, by keeping track of the user attention on a ledger system implemented in Brave, while always maintaining users' anonymity. User attention, as a very valuable asset, is not being rewarded correctly and users do not get anything while they navigate webpages and see the ads. The solution proposes that users also start receiving rewards for time spent seeing the ads while navigating, based on the amount of time they spend looking at them. \\

In order to reduce fraud, they propose approaches - calling them basic attention metrics (BAM) - to correctly identify users paying attention to ads. When the user attention is identified, it is saved in an anonymous way, while also ensuring that users do not get rewarded by paying attention to the same ad more that once. They define a \textit{proof-of-attention}, which ensures that a user can only see and get attributed to an ad once and maintains users' anonymity. This is achieved by using \textsf{ANONIZE} \textbf{[REF ANONYZE PAPER]} algorithm in a first stage. According to the authors, the algorithm is "the first implementation of a provably-secure multiparty protocol that scales to handle millions of users".  The BAT team says that they may also invest into using algorithms such as BOLT \textbf{[REF BOLT PAPER]}, zkSNARKs \textbf{[REF SNARKs PAPERS]} and STARKs \textbf{[REF STARKs PAPER]} to protect users' privacy.

\subsubsection{Kin}



\subsubsection{Monetha}


\subsection{Projects Contributions}
\subsubsection{Advertising}

\subsubsection{IAB}

\subsubsection{Developers Reputation}






