\section{Blockchain Limitations and Proposed Approach}
\label{sec:limitations}


%The current status of blockchain developments, both Bitcoin and Ethereum, can limit the development of projects that depend on the technology to support the required business use cases. There are three use cases which the AppCoin protocol tries to solve in order to enable a fully working AppCoin economy:
With the current position of the blockchain industry, business use cases are reliant on technology supported by Bitcoin and Ethereum, which can delay the development of projects. There are three use cases which the AppCoin protocol tries to solve in order to enable a fully working AppCoin economy:

% \subsection{Use Cases}

\begin{itemize}
    \item Apps advertising in app store
    \item In-App Billing 
    \item Developer reputation leading to approval of their apps
\end{itemize}

From the aforementioned use cases a list of requirements on the technology can be established. 

\begin{itemize}
    \item Store digital currency / value securely
    \item Couple value storage and their transfers with logic
    \item Scalability for millions of users
\end{itemize}

The first requirement is solved by using a blockchain technology. Either Bitcoin or Ethereum work for this. For the second requirement only a type of blockchain technology supporting smart contracts will work, i.e. Ethereum, Nxt or Tezos.

\medskip

Unfortunately, neither basic Ethereum nor any other current blockchain for that matter are able to scale for the needs of an app marketplace. This problem is worsened further considering a platform that should potentially work for all the app marketplaces.

\medskip

First let us have a detailed look at the challenges in question.

\subsection{Blockchain limits}

Let us imagine a scenario of a working AppCoin economy using a blockchain technology like Ethereum. At the present there are about 2 billion Android users. The average Ethereum transaction is around 160 bytes \cite{EthereumTransactions}. If we wanted to create an AppCoin economy directly on Ethereum using these numbers, we would get a system with the characteristics shown in Table \ref{table:ethereumscalability}.

\begin{table}[!htbp]
\centering
\begin{tabular}{|l|r|}
\hline
Users              & 2 000 000 000    \\ \hline
Avg. daily TX      & 1                \\ \hline
Avg. TX size       & 160 bytes        \\ \hline
Daily TX           & 2 000 000 000    \\ \hline
TX per second      & 23 148           \\ \hline
Daily TX size      & 298.02 GB        \\ \hline
TX size per second & 3.53 MB          \\ \hline
TX size per month  & 8.73 TB          \\ \hline
TX size per year   & 104.77 TB        \\ \hline
\end{tabular}
\caption{Scalability requirements for AppCoin economy}
\label{table:ethereumscalability}
\end{table}

\medskip

{\bf Transaction Volume}

Scalability in terms of volume is the first problem of Ethereum. At present it handles only up to 20 transactions per second \cite{eth_scaling}. That is nowhere near the required 23 000 average transactions per second. At peak times the system should support a number of transactions several orders of magnitudes higher; for example VISA normally handles around 1667 transactions per second and at peak times it claims it can complete around 56 000 transactions per second \cite{eth_scaling}.

\medskip

{\bf Fees}

The topic of fees is closely connected to the possibility to support micro-transactions - transfers of value in the range of a few cents. At the time of writing the cost for one transaction in Ethereum is around 10 cents of a US Dollar for a transaction that takes around 1 minute to execute \cite{ethgasstation}. If the user is willing to wait around 10 minutes, the cost is 1 cent. This is both too slow and too expensive to facilitate micro-transactions.

\medskip

{\bf Latency}

As mentioned in the previous section, in Ethereum, one has to wait until a transaction is part of the blockchain. To have a reasonable guarantee that the transaction is part of the main blockchain and not in one of a number of possible branches, one needs to wait until 7 blocks have been created. This is because two miners could submit the next block at the same time and it will take some time to decide whose block will continue the chain. While this might not be an issue for high value money transfers in the AppCoin economy the transactions have to be executed within few seconds or faster (preferably instantly) in order to enable meaningful IAB and user experience.

\subsection{Existing technology}

Scalability restrictions are present in any blockchain technology using mining by \textit{proof-of-work}, such as Bitcoin and Ethereum. For Bitcoin - the oldest blockchain implementation - the proposed solution is the Lightning Network \cite{LighthingNetwork}.

%why is the lightning network the only solution referred? should talk about proof-of-stake?

\subsection{Ethereum and Bitcoin based}

Effectively there is one possibility for how to tackle the scalability problem with Bitcoin and two solutions with Ethereum. The first one is keeping the proof-of-work based blockchain and enhancing it with a direct-channel-payment solution like the Lightning Network. For Ethereum there are two major technologies that are going to be introduced also in this section.

\subsection{Lightning Network}
In brief the Lightning Network allows for creation of bidirectional payment channels that are off-chain \cite{LighthingNetwork}. Using such a channel two parties agree to deposit money into a common entity called a channel and this information is stored on the blockchain. There is a 2-signature entry for the channel with the total sum of the deposit saved. From this moment on the two parties can exchange between them any number of payments in the form of signed receipts. The only condition is that the balance for one party doesn't exceed the total of the deposit. Should a case like this happen, the parties can increase the deposit. These off-chain transactions are signed and sent direct. They are not broadcast. In contrast all the transactions on the main blockchain need to be broadcast to all participants. 

\medskip

This approach solves the blockchain scalability problem, such as:

\begin{itemize}
    \item The messages are direct and not broadcast. The volume of transactions between the two parties is only limited by their processing capabilities. Also, compared to blockchain, not all the transactions have to be stored but only the latest receipts, i.e. the latest balance for both users, must be stored instead of all transactions. This solves the problem of storing huge amounts of old transactional data that is inherent in blockchain.
    \item The only fees are for opening and closing the payment channel on the blockchain. For payments over the payment channel there are no extra fees. Once it is established micro transactions are possible without further limitations.
    \item Finally by using direct peer-to-peer (P2P) communication, transactions can be exchanged as fast as the underlying network between the two parties will enable it. Thus the problem of latency is solved.
\end{itemize}

Thus the original blockchain scalability limitations are resolved for the case of two directly connected parties. Of course having open channels between all the participants in the AppCoin economy is both not feasible and economically viable (because of opening and settling fees on the blockchain). Therefore the Lightning Network proposes a solution, where the payment channels are stored in a graph and payment can be routed securely keeping the above mentioned characteristics among multiple parties. This is enabled using hashlocks. For more detailed information, please have a look at either the specification of Lightning Network or the excellent introductory article \cite{starkness}.



\subsubsection{Raiden}
Raiden network is an early implementation of the Lightning Network protocol for Ethereum. Therefore, it is an off-chain scaling solution for Ethereum. It promises near-instant, low-fee and scalable payments and enables micro transactions \cite{Raiden}. Raiden runs as a network of nodes establishing payment channels among users. It uses locked funds in a smart contract in Ethereum and payments happen inside of Raiden until one party chooses to settle. Raiden will work for any Ethereum tokens implementing the ERC20 standard.

\medskip

One of the key characteristics of Raiden is its maturity. Currently it is the only known technology for Ethereum with existing implementation (i.e. beyond white paper stage) enabling a working App Coin economy. It has been under development for the past two years and it is reaching a Minimal Viable Product stage, currently being in a developer preview stage.

\medskip

Privacy is another key characteristic of Raiden.  All the transactions are known only to the involved parties and not public knowledge like in Ethereum.

\medskip

Finally the maturity can also be seen as a weak point for Raiden. Even though it is beyond proof-of-concept stage it is not recommended to be used on the live Ethereum network by its creators.

\subsubsection{Plasma}
Plasma is defined as a platform for "scalable autonomous smart contracts" \cite{Plasma} and it is another attempt to bring the Lightning Network technology from Bitcoin to the Ethereum world. This is achieved by means of a tree of hierarchical chains where most of the transactions can be settled in the sub-chains and do not need to go to the expensive and slow main chain. The main chain will be needed only to settle disputes coming from sub chains and would serve as final validation.

\medskip

A strong point for Plasma is that it is backed by key people from both Ethereum and Lightning Network, i.e. Vitalik Buterin and Joseph Poon. Another aspect supporting Plasma is that it has been endorsed by OmiseGO to be used for their decentralised exchange and payment platform\cite{omisego_plasma}. What speaks against Plasma at this time is the fact that the project has only a white paper, without any working code base yet.

\subsubsection{OmiseGO}
OmiseGO is an Asian payment gateway. It plans to use Ethereum and Plasma as a foundation for creating a decentralised exchange and payment platform \cite{OMG} and the company intends to launch its own blockchain called OMG that will be complementary to Ethereum. Ether will be then used as a medium of interchange for various assets being exchanged on this blockchain.

\medskip

Here are some of the strong points of OmiseGO:
\begin{itemize}
    \item Consensus rules for high performance activity;
    \item Rapid execution and clearing. Proof-of-Stake;
    \item Ability to handle extremely high volumes of transactions with final delivery in Ethereum;
    \item Availability as white label e-Wallet solution
\end{itemize}

The weak points include:
\begin{itemize}
    \item The project is at a very early stage. It was only announced at the beginning of August 2017. Currently, there is only a white paper available.
    \item As a result of the coupling with Ethereum, the OmiseGO white paper recommends to validate transactions simultaneously on the Ethereum blockchain for maximum security.
\end{itemize}

\subsection{Independent blockchains}
The second solution to achieve global consensus on scalability for blockchain is the use of a proof-of-stake as a block minting algorithm. For Ethereum this would be the Casper protocol which has been commonly talked about in the past years in the Ethereum community but is not implemented yet. As Casper is not in use yet we will examine here other projects that have implemented minting instead of mining blocks. A brief look will be also given to IOTA, which solves the scalability problem with a directed acyclic graph called Tangle instead of blockchain.

\subsubsection{Nxt}
One of solutions not based upon Ethereum is Nxt. It is described as "an open source cryptocurrency and payment network launched in November 2013 by anonymous software developer BCNext" \cite{Nxt}. Compared to Ethereum one can see Nxt as a monolith trying to include major features from all coins. It describes itself as an modern economic system based on cryptography and blockchain technology. On the other hand Ethereum built a low platform and programming language in which one can program generic contracts. Other major differentiation of Nxt to Ethereum is the fact that it is based on proof-of-stake for creating new blocks.

\medskip

Nxt is unsuitable for creating an App Coin Economy as it targets a different scenario. Focusing on managing a business, assets and customers and interaction between them are recorded in a secure way instead of a blockchain backed technology enabling users to create generic decentralised apps. Another point against Nxt is the fact that its beginning was obscure. 73 anonymous accounts received stakes in exchange for Bitcoin accounts and it is these people who can mint new blocks. And also that there has not been much buzz around and development for Nxt recently.

\subsubsection{Tezos}
Compared to Nxt the next examined project has received a great deal of attention recently and finished its ICO having raised more than \$200 million.

Tezos describes its ultimate aim to create blockchain technology working better than both Ethereum and Bitcoin. It wants to provide users with financial incentives for maintaining on-chain consensus mechanism. The main differences to Ethereum are:

\begin{itemize}
    \item Built-in consensus mechanism for governance.
    \item Decoupled from Ethereum. Separate blockchain with proof-of-stake consensus for block minting.
    \item OCaml as programming language for creating smart contracts.
    \item Design-wise Ethereum is a generic and low-level solution while Tezos aims to provide a fat protocol with a lot of features \cite{TezosEth}, not unlike Nxt.
\end{itemize}

\subsubsection{IOTA}
The goal of IOTA is to create a lightweight network that allows a machine economy. A machine economy is a situation where IoT (Internet of Things) connected devices communicate and execute payments between each other. IOTA does not use blockchain but a directed acyclic graph called Tangle. Currently it is in Beta stage with a reference implementation. One of the major advertised features is that there are no fees. In Tangle, each new transaction confirms at least 2 previous transactions; that is the "fee" to participate. This might delay the transaction, especially if the network is small. There are no miners. The IOTA white paper states it is quantum proof (invulnerable to encryption-breaking attacks), that it has a high level of supply and the tokens are not divisible, unlike Bitcoins or Ether. For a primer on IOTA see \cite{IOTA}.

\subsection{Proposed Approach}


\begin{figure}[!ht]
\centering
\includegraphics[width=\textwidth]{diagrams/offchain_wallets.eps}
\caption{On-chain and off-chain transactions.}
\label{fig:offchain}
\end{figure}


As seen in the previous section, there are blockchain currencies that can securely store digital value. But as we have seen, doing transactions over them does not scale, is slow and requires fees. Therefore alternatives like direct payment channels have to be evaluated.

\medskip

In general terms, the App Store acts as the gateway to the digital payment world for the user. He deposits his valuables (preferably Ether) with the store in a payment channel using a smart contract. The same happens for developers. And from this moment on they can utilise the micro-payments within the App Infrastructure. The payments are instant, cheap and there can be any number of them.

\medskip

The user has to pay fees on each Ether deposit to his account with the App Store and on checking out, i.e. settling his payment channel. Further fees may be the taxation done by the App Store used to pay for its infrastructure.  If there are several App Stores in the same ecosystem like we envision, there might be further (minimal) fees to forward the payments from one to another.

\medskip

Figure \ref{fig:offchain} depicts a schema of the wallets on-chain (Ethereum network) and the wallets off-chain using Raiden or OmiseGO.

% TODO explain better the division

\medskip

Returning to the requirements from the beginning of this chapter, we plan to use Ethereum for storing value and smart contracts. To have scalability, Raiden is being evaluated for the first Beta version of the reference implementation of the protocol that is due six months after the ICO. %could not understand sentence

\medskip

The choice for Raiden is because - at the time of writing - it is the only platform that has a test network and can be used. Yet OmiseGO has a promising outlook and in the future could be adopted depending on its progress. The company will be kept under close monitoring as it might develop the potential to substitute Raiden.




