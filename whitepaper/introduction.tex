\section{Introduction and Problem Statement}
\label{sec: introduction}

\subsection{Motivation}

In 2016, the Aptoide app store was used to install 1.1 bn apps in 200 million unique devices\footnote{Aptoide web site is considered the \#692 most visited site in the world by SimilarWeb (Sept 2017).}. Although the Aptoide brand may - or not - be familiar to the reader before reading this white paper, one out of five young people between the ages of 16 and 25 worldwide uses Aptoide. In certain countries - like Brazil or Mexico\footnote{Aptoide Top 5 countries: Brazil, Mexico, US, India, Italy} - that number increases to one out of three.

\medskip

Over this incredible journey to earn the trust of Aptoide users, without any paid acquisition, we discovered that app stores can be much more. The app discovery can be much better. The financial transactions - like advertising and in-app purchases - can be much more efficient. The sharing can be much more powerful. The current state does not benefit enough the developers or the user. It only benefits Google and Apple. In a closed market, they can impose margins and their own distribution rules.

\medskip

The proposed protocol in this document is a call to the community. It is a call to developers, to other app stores, and to users. It is a call to work together towards a free-entry app store market. A market that unlocks the world of in-app purchases to billions of users. A market that benefits the talented developers providing them significant revenue and transparent ways to reach their users. AppCoins envisions a world where app stores compete by innovation and service level.

\medskip

The same way open source and community helped Aptoide to reach 100 million users in 4 years, we are certain that {\em blockchain} is the technology that will enable this revolution, providing trust and transparency. If you share this vision, join us on this journey \cite{lucas}, collaborating in the protocol, developing open source software and spreading the word. Changing the app store ecosystem and breaking monopolies.

\medskip

\subsection{Historical perspective}

App stores are a distribution channel between the developer and the end user. Although software distribution exists since there is software development, the current model of smartphone became popular with the launch of Apple's App Store in July 2008 and with its pre-load in iPhone 3G. In the same year, but later in August 2008, Google announced the launch of Android Market \cite{wiki:market}, the app store for Android.

\medskip

%These initial app stores followed a centralised model where one entity is responsible for assuring the core features of software distribution: file delivery, app discovery, financial transactions and app approval. This centralisation in the context of app stores means no transparency and being closed source. As the smartphone user base grew, the centralised model started to show severe flaws. The flaws and problems identified are strongly related with the existent model: a lack of trust and economical efficiency.
These initial app stores followed a centralised model where one entity is responsible for assuring the core features of software distribution: file delivery, app discovery, financial transactions and app approval. This centralisation comes with its inherent drawbacks such as little to none transparency and being closed source. As the smartphone user base grew, the centralised model started to reveal additional flaws: a lack of trust and economic inefficiencies.

\medskip

%Non-transparency means app stores can avoid disclosing the reasoning behind their decisions. An example is censoring apps that are seen by them as competitors or that do not follow the rules set by them. On the other hand, being closed source creates the possibility to hide processes from users, such as collecting user data (e.g. user age, preferences, apps installed, etc) for other purposes. By not being transparent, app stores do not earn the trust among the different stakeholders: developers, advertisers, users and OEM manufacturers. By being centralised, they cannot benefit from a shared and crowdsourcing economy. Being closed source and hiding data, they do not promote competition and innovation.
Another weakness of the centralised app economy is the intransparency behind rule-making and enforcement. For instance, apps are commonly censored if they compete with the app store owner's economic interests. In addition, collected personal user data (e.g. user age, preferences, apps installed, etc) can be exploited for other purposes. A lack of transparency erodes trust among the participating stakeholders: developers, advertisers, users and Original Equipment Manufacturers (OEM). Moreover, the central authority behind the app store reaps a disproportional high share of the created revenue in the smartphone value chain. Further, centralisation also stifles competition and innovation.

\medskip

%The AppCoins protocol covers three core app store use cases:
The AppCoins network redefines the following three app store core processes:

\begin{itemize}
%\item {\bf Advertising inside the app store}: The transactional flow where a developer pays for a user to install their app or game. There are different advertising models depending on the action that triggers the actual payment of the Ad: CPI (Cost per Installation), CPA (Cost per Action), CPM (Cost per Thousand Impressions) and others. There are different technologies and platforms to support it: Ad networks, Exchanges and RTB (Real Time Bidding).
\item {\bf Advertising inside the app store}: Developers advertising to users to install their app or game. There are different advertising models depending on the intended action: CPI (Cost per Installation), CPA (Cost per Action), CPM (Cost per Thousand Impressions) and others. There are different technologies and platforms to support it: Ad networks, Exchanges and RTB (Real Time Bidding).
%\item {\bf In-App Purchase}: When there is something that the user wants to buy inside the app or game, like gems or unlock levels, the purchase mechanism is done through the app store. To enable those transactions, the developer has to integrate the SDK from the app store or to use the app store API.
\item {\bf In-App Purchase (IAP)}: When users want to unlock premium features inside the app or game, the purchase mechanism is tied to the respective app store. To enable payment transactions, the developer has to either integrate the SDK from the app store or use its API.
%\item {\bf App approval}: In order for the app to be available, the developer has to go through an approval process where the app store screens the app through automatic tools like anti-virus, anti-malware tools and static and dynamic code analysis platforms. Some app stores also manually test the app.
\item {\bf App approval}: To offer the app in the store, developers have to go through a stringent approval process in which the submitted app is screened by anti-virus and anti-malware tools, as well as static and dynamic code analysis platforms. Some app stores also rely on manual app testing.
\end{itemize}



Not only Developers and OEM manufacturers, will benefit of app stores supporting the AppCoins protocol.

Users will benefit as well: 1) easier to transfer money between users in a peer-to-peer way 2) possibility of earning coins by installing and giving attention to a game or even evaluating the quality and safety of an app 3) be censorship resistant and avoiding the censorship potential of a centralized approach.  

%In the next sub-sections, we'll analyse each of the above flows and the main problems faced today.
In the next subsections, each of the flows and problems are analysed.

\subsection{Advertising}
\label{subsec:intro_ads}


%Currently, the three flows presented in Figure \ref{fig:exist_flows} do not have any interaction among them. They are isolated and handled by different app store teams. The resources and information generated by one flow are not reused by the others. The intermediaries are many and were introduced to solve the lack of trust and the need to integrate with different players in a fragmented market.  In the next sub-sections, we will analyse each of the flows individually and the main problems faced today.
The flows presented before are not interacting with each other. Resources and information generated by each flow are siloed from each other. The numerous intermediaries mushroomed to address the lack of trust and integrate the various players in the fragmented market. In the following subsections, we will analyse each of the flows individually and the main problems faced today.

% XXX which flows: the flows cannot be easily identified
% XXX The resources ... sentence seems to be redundant.

\medskip

For a developer or a publisher, the most natural place to advertise an application or game is where the users are looking for that kind of content: the app store.

\begin{figure}[!ht]
\centering
%\includegraphics[width=11cm]{diagrams/current_flows.png}
\includegraphics[width=\textwidth]{diagrams/cpi_flow.eps}
\caption{Cost per Installation (CPI) ecosystem.}
\label{fig:cpi}
\end{figure}

For simplicity, we will focus on the Cost per Install (CPI) model shown in Figure \ref{fig:cpi}, since the difference to the other models, like Cost per Action (CPA) or Cost per Thousand Impressions (CPM), is a matter of who shares the risk and captures the value.

In figure \ref{fig:cpi} the different blocks of the industry are depicted. Demand-Side Platform (DSP), Supply-Side Platform (SSP), Real-Time Bidding (RTB), Data Management Platform (DMP) are different platforms that help the publishers and the advertiser to maximize the impact of their business. 

In an advertising model where the advertiser (developer) bids for an installation, we have three different moments:

\begin{itemize}
\item {\em Campaign creation}: The developer (advertiser) defines the conditions for the ad to run in the store. Typically, he establishes a value for the bid representing the value that he is willing to pay for an install. There are other types of conditions called ``filters'', which represent target requirements. For example, requirements stating that the campaign must run in a specific country, a specific smartphone, a specific operating system version, and others.
\item {\em Impression}: When the campaign conditions are met and the bid is competitive, the ad is shown. The user may click on the ad to see the complete description of the app.
\item {\em Install}: If the user installs the app, thus converting the impression, an attribution is due and the corresponding money is transferred.
\end{itemize}

% Create a itemize list above 

At each of the above moments, the lack of trust between the developer and the user carries different risks. Table \ref{tab:risks} summarises the different risks at each stage.

% table

\begin{table}[ht]
\centering
\begin{tabular}{|l||l|l|l|} \hline
{\bf Role} & {\bf Campaign} & {\bf Impression}  & {\bf Install} \\ \hline
{\bf User} & & & Is not a real user \\ 
 & & & ({\em R1: Risk of fake person}) \\ 
 & & & Double conversion  \\
 & & & ({\em R2: Risk of double attribution}) \\
 & & & Don't use the app  \\
 & & & ({\em R3: Risk of no attention}) \\  \hline
{\bf Publisher  /}  & & & Selling the data to third-parties \\ 
{\bf App store} & & & ({\em R4: Risk of data leak})\\ \hline
{\bf Developer} & Not enough funds & Run out of & Don't pay \\  
 & to start campaign & budget & the conversion \\  
  & ({\em R5: Risk of default}) & ({\em R5: Risk of default}) & ({\em R6: Risk of repudiation}) \\  
\hline\end{tabular}
\caption{Risks in advertising industry classified by action and by role.}
\label{tab:risks}
\end{table}

The risks presented above are today managed in different ways by the advertising ecosystem and have different impacts:

\begin{tcolorbox}[enhanced jigsaw,sharp corners, drop fuzzy shadow=ShadowColor]

The {\bf\em R1.1: Risk of fake person} consists of the impression of the ad and later installation being presented to a non-real person (bot,...) with the purpose of deluding the advertiser. 


The {\bf\em R1.2: Risk of double attribution} happens with the possibility of the same user to count twice as a conversion, leading the developer to pay two times what was due.


The {\bf\em R1.3: Risk of no attention} consists in the user installing the app that is being advertised but paying no attention to it. Even if they open the app, there is the possibility of no interaction with the app, i.e. the user opens and immediately closes the app. This leads to a zero return-of-investment.


The {\bf\em R1.4: Risk of data leak} consists in the information regarding the user being leaked to third-parties for advertising purposes. Information about the user's preferences are aggregated in Data Management Platforms (DMP) and later used by advertisers in programmatic / RTB targeting. 

The {\bf\em R1.5: Risk of default} consists in the developer creating a campaign but not having enough funds to pay the conversions that are generated in that campaign, leading to him not paying the due amount.


The {\bf\em R1.6: Risk of repudiation} happens when the developer does not recognise the installation, failing to attribute the conversion to the publisher. The attribution is generally monitored by tracking platforms like AppsFlyer, Adjust or Kochava that have multiple variables that can be changed by the developer to define what it considers a real attribution. These variables can take in account the time window period between the click URL and the conversion, the network fingerprint, among others. Attribution, or the lack of it, is harming the industry with only 15\% to 25\% of the real installations being considered conversions\footnote{Values based on Aptoide experience.}.

%XXX what does mining the industry mean?
%XXX provide reference for the 15-25% statement

\end{tcolorbox}

These risks in the Advertising flow will be considered in a section ahead in the design of the AppCoins blockchain.


\subsection{In-App Billing}
\label{subsec:intro_iab}

% Description of In-App Billing inside the stores

In-App Billing (IAB), also called In-App Purchase, consists in the possibility for the user to buy digital items inside an app or a game. Although those items are perceived to be bought inside the App, the items are bought through the app store.

\begin{figure}[!ht]
\centering
\includegraphics[width=\textwidth]{diagrams/iab_flow.eps}
\caption{Current IAB flow and intermediaries.}
\label{fig:iab_flow}
\end{figure}


The need for the transactions to go through the app store were introduced as mandatory by Apple App Store and then by Google Play. The conditions for the developer's app to be distributed is that all the financial transactions have to be managed by the app store. 

\medskip

The app store adds some value to this flow: 1) it may know the customer already and have their payment data, thus easing the entrance hurdles for the user and providing a better user experience, 2) it has the trust of the user when the developer may not have it yet and 3) it develops the necessary technology, allowing the developer to focus on the app development.

\medskip

Although IAB represents a market with a huge volume of transactions processed by Google Play and Apple, there are still two big challenges.

%XXX provide source for "huge volume of transactions by ..."

\medskip

The number of users with a credit card loaded in the store is still a minority. Only small part of the world population has access to credit card. Alternative methods like pre-paid cards are an approach but they are physical and depend on points of sale, therefore do not scale well.

%XXX provide source for "Only small part of the world population has access to credit card."

\medskip

On the other hand, some of other payment methods like carrier billing have prohibitive margins that compromise the revenue share of 70\% for the developer. In some markets, the margin of the telecom operator varies between from 35\% to 60\% of the cost of the transaction. The reasons given by the telecom operators are: 1) high risk of fraud that has to be compensated and 2) the users may cannibalise the telecommunications balance so the margin has to pay that possibility.

%XXX provide source for margin being between 35% and 60%

\medskip

Providing the user has a proper payment method, there are still some risks that have to be mitigated:

\begin{tcolorbox}[enhanced jigsaw,sharp corners, drop fuzzy shadow=ShadowColor]

The {\bf\em R2.1: Risk of user data leak} consists in the information regarding the user being leaked to third-parties for advertising purposes. Information about the user preferences are aggregated in DMP platforms and later used by advertisers in programmatic / RTB targeting.

%XXX R2.1 is copy and paste of R1.4!

The {\bf\em R2.2: Risk of digital goods lost} may happen when a user buys a digital good inside the game or app but it is not delivered. Often, the user does not have a way to recover the payment or claim the digital good.

The {\bf\em R2.3: Risk of double payment} occurs when the user pays twice for the same in-app item purchase. Also in this case, the user may not have a proof that they paid twice.

The {\bf\em R2.4: Risk of digital items cloning} when the user is able to duplicate and transfer the digital good to another user, leading to losses for the developer that charge once for a digital good that is used twice.

\end{tcolorbox}

A platform that handles the IAB transactions has to deal with those risks.


\subsection{App Approval}
\label{subsec:intro_approval}

% Description of today state: central approval, manual QA, automatic QA, time taken, arbitrary approval

The app approval is one of the more critical challenges of an app store. By definition, the app store is a channel between the developer and the user.

\begin{figure}[!ht]
\centering
\includegraphics[width=\textwidth]{diagrams/apps_approval_flow.eps}
\caption{App approval in centralised App Stores.}
\label{fig:app_approval_flow}
\end{figure}

% include contribution from João Carneiro
In order to enforce security, legal and business requirements, app stores define limits in terms of acceptable app behaviour and/or content. These policies also mirror the store's philosophy (e.g. defining the acceptable content) and protect both users and developers against unwanted or potentially dangerous behaviour, thus promoting trust. Policies can include general categories such as safety - protecting against malware behaviour, offensive content or physical harm - or legal - protecting privacy and intellectual property. More restrictive stores such as the Apple App Store also imposes strict rules regarding the user interface design, minimum functionality and quality. \cite{GooglePolicyWebsite, ApplePolicyWebsite}

%XXX I don't get the meaning of "general categories"

The risk of infringement occurs when new apps are added to the store. Therefore, stores which are open to public upload of apps (e.g. Google Play Store or Apple Play Store allow submission by developers) need to ensure that uploaded apps abide by their rules by putting them through a reviewing process. 

\medskip

The app screening may be performed through manual and/or automatic processes and differ between stores as they are defined by their own policies. The manual process involves a group of people (typically belonging to the Quality Assurance and/or the Security Team) who manually install and test apps on real devices. They examine the apps' behaviour and content in order to decide whether each app respects the store's policy. The automatic process consists of a computer program which automatically analyses the submitted apps and compares features to a given dataset of rules, signatures, unwanted apps, content or behaviour. Multiple techniques may be used by the program to automatically classify given apps into unwanted, accepted or unknown states \cite{Bhattacharya2017}.

\medskip

Google's Play Store and Apple's App Store, the current largest and most well-known app stores, use a combination of both processes. When a new app is submitted to their store, they first go through an automatic process which will automatically discard identified unwanted apps and then proceed to the manual process. However, the two stores differ in the techniques they use in their automatic processes and the amount of apps that go through manual reviewing \cite{AppleInsiderWebsite, AndroidWhitePaper}.

\medskip

Apple App Store approval flow is simple. All submitted apps go through an automatic static analysis process, a method which examines the app code without running it. In this process, the apps are analysed for traces of calls to Apple's private API as the company's policy only allows calls to their public API. The identified apps are discarded while all the remaining apps are passed on for manual review \cite{AppleInsiderWebsite, AppleApprovalFortune}. Apple states the following most common reasons for failing their strict manual reviewing process: crashes and bugs, broken links, placeholder content, incomplete information, inaccurate description, misleading users, substandard user interface, advertisements, web clipping, similar apps and not enough lasting value \cite{AppleReviewRejections}. According to Apple, the complete review process takes on average between 24 (50\%) to 48 hours (90\% of submitted apps) \cite{AppleReviewTime}. 

\medskip

Google's Play Store has a more evolved reviewing system where the automatic process involves a complex machine learning engine. This engine relies on multiple technologies including static and dynamic analysis (where both code and runtime behaviour is analysed), heuristic and similarity analysis (for finding new trends of unwanted apps) and signatures (identifying known unwanted apps). The engine also includes features from external independent security research as well as the developer's behaviour (history with other apps and billing profile) as well as metadata such as ratings and downloads. The automatic process assigns to each application a risk level ranging from safe to harmful. Low risk applications are automatically accepted and high risk applications are automatically rejected. Apps with medium risk level are submitted for manual review \cite{AndroidWhitePaper}. 

\medskip

Although these app approval systems are capable of detecting a large number of unwanted apps, they also pose problems to developers. Apple's automatic process has shown problems with false positives and rejecting legitimate apps \cite{AppleInsiderWebsite} and their strict policy is known to frequently change the categories of rejected apps, posing problems to developers of such apps. Also, Apple's reviewing process strongly based on human analysis is known to have flaws, namely not being able to detect apps which hide their malware behaviour by being inactive for a given amount of time and showing a regular behaviour in order to escape the human test \cite{AppleFlaws1}. Other reports \cite{AppleFlaws2} have also shown a big presence of scamware in Apple's store where apps are able to scam users into paying for unneeded services. Google's more automated system has also been shown to have flaws \cite{AppleApprovalFortune} due to its more permissive system with apps being accepted without manual evaluation. Frequent security reports show breaches in the security control of Play Store reporting the existence of multiple malware (ransomware, backdoor and trojans) infected apps compromising several millions of devices and thus posing serious threats to users \cite{GoogleMalware1, GoogleMalware2, GoogleMalware3}.

\medskip

Both Apple's App Store and Google's Play Store have a history of refusing and banning apps. Examples of recent complaints include the rejection of the social network GAB's app by both Apple and Google \cite{AppRefusedGAB}, the refusal of music streaming service Spotify's app update \cite{AppleRefuseSpotify} or the Anti-Spam App \cite{AppleRefuseTRIAD} by Apple and the rejection of Popcorn Time, TubeMate, Adguard or Fildo by Google \cite{GoogleBannedApps}. However, these rejections are often considered unfair by developers who claim an abusive and anticompetitive behaviour as the apps conflict with other services provided by the app stores and have motivated a number of complaints to legal authorities \cite{AntiCompetitiveClaim}. \\

\medskip

As described above, several risks can be found from the current centralised app approval processes:

\begin{tcolorbox}[enhanced jigsaw,sharp corners, drop fuzzy shadow=ShadowColor]

The {\bf\em R3.1: Risk of Malware} is the possibility of an app or game submitted to the app store being infected with a virus or malware that steals data or damages the user's device.

The {\bf\em R3.2: Risk of user data leak} consists in the information regarding the user being leaked to third-parties for advertising purposes. Information about the user's preferences are aggregated in DMP platforms and later used by advertisers in programmatic / RTB targeting.

%XXX again, R3.2 is just a copy and paste of R1.4

The {\bf\em R3.3: Risk of censorship} when an app store blocks the publishing of an app or game based on political, religious or social factors that are subjective and totally unrelated with technical aspects. The censorship can be self-inflicted when the company running the app store follows orders or guidelines of national governments or can be result of technical external restrictions that limits the access to the app store (The Great Firewall of China, for example).

The {\bf\em R3.4: Risk of arbitrary decisions} happens when the app store denies the distribution of an app based on ``anti-competition clauses'' \cite{PlayTermsService} or other reasons only related to its business interest, even if the interest is in other markets or industries.

\end{tcolorbox}

\subsection{Paper organisation}

This paper is organised in the following chapters. In this chapter we started to introduce the flows, the current challenges and the flaws they carry. 

\medskip

Chapter \ref{sec:design} will propose the overall design of the solution for the core app store flows supported by blockchain technology.

\medskip

Chapter \ref{sec:protocol} will dive deep in the blockchain technology, presenting the main data structures and algorithms that are proposed.

\medskip

The current limitations of the blockchain technology when applied to app stores are introduced in Chapter \ref{sec:limitations}.

\medskip

In Chapter \ref{sec:related}, related work that shares common approaches with the AppCoins protocol are introduced, as well as projects that inspired parts of the AppCoins protocol.

\medskip

The future protocol developments will be included in Chapter 6 and this document will end with acknowledging the contributions of the several community members that contributed to this document with their suggestions and opinion.


% TODO replacing 6 by the \ref tag

